\chapter{Energy Dynamics Formalism}
\label{c_en_formalism}

In the last section of the previous chapter, I analyzed the experimental and simulated turbulence using simple and common statistical methods. Never did I assume any kind of model for the turbulence,
nor did I take full advantage of the wealth of spatial information provided by the simulations. In the remaining chapters, I do use the simulated physics model along with the turbulent spatial
structures to analyze the nature of the turbulence from an energy dynamics perspective. The energy dynamics provide direct information about energy injection into the turbulence from the equilibrium
gradients, energy transfer among different fields and between different normal modes, and turbulent energy dissipation. This information allowed me to uncover the mysterious mechanism that
drives the $n=0$ fluctuations so strongly in the simulations (see Figs.~\ref{n_statistics} f) and~\ref{phi_statistics} f)). The mechanism is a nonlinear instability. I will provide evidence
for this in the next chapter, but in this chapter, I will derive the dynamical energy equations and explain what they mean.

\section{Total Energy and Dynamics}
\label{s_tot_en_dyn}

First, I consider the total, volume-averaged energy and energy dynamics. The total volume-averaged energy of the fluctuations (in normalized units) is:

\beq
\label{tot_energy}
E = \frac{1}{2} \int_V  \left[ P_0 \left((N/N_0)^2 + \frac{3}{2} (T_e/T_{e0})^2 \right) + N_0 \left( \frac{m_e}{m_i} \vpe^2 + (\gradperp \phi)^2 \right) \right] dV,
\eeq
where $P_0 = N_0 T_{e0}$ is the equilibrium pressure.
The $\frac{1}{2} P_0 (N/N_0)^2$ term is the potential energy due to density fluctuations, $\frac{3}{4} P_0 (T_e/T_{e0})^2$ is the electron temperature fluctuation potential energy,
$\frac{1}{2} N_0 \frac{m_e}{m_i} \vpe^2$ is the parallel electron kinetic energy, and $\frac{1}{2} N_0 (\gradperp \phi)^2$ is the ${\bf E \times B}$ perpendicular kinetic energy.
The energy contained in the electric field is smaller than the perpendicular kinetic energy by a factor of $(v_A/c)^2$ and is therefore neglected.

The dynamical energy evolution $\partial E/\partial t$ can be obtained using Eqs.~\ref{ni_eq}-~\ref{te_eq} in the following way. 
First, take Eq.~\ref{ni_eq} and multiply both sides by $\frac{T_{e0}}{N_0} N$ and integrate over the volume.
The result is:

\beq
\label{Nsq_en_ev}
\pdiff{E_N}{t} = \left< - T_{e0} N {\mathbf v_E} \cdot \grad {\rm ln} N_0 - T_{e0} N \gradpar \vpe - \mu_N \frac{T_{e0}}{N_0} (\gradperp N)^2 + \frac{T_{e0}}{N_0} N S_N \right>,
\eeq
where $E_N = \frac{1}{2} \left< P_0 (N/N_0)^2 \right>$ with $\left< \right>$ shorthand for the volume integral $\int_V dV$. 
Next, multiply Eq.~\ref{ve_eq} by $N_0 \frac{m_e}{m_i} \vpe$, Eq.~\ref{rho_eq} by $- \phi$, and Eq.~\ref{te_eq}
by $\frac{3}{2} \frac{N_0}{T_{e0}} T_e$ and volume integrate, giving:

\beqar
\label{vesq_en_ev}
\pdiff{E_v}{t} & = &  \left< - T_{e0} \vpe \gradpar N - 1.71 N_0 \vpe \gradpar T_e  + N_0 \vpe \gradpar \phi - \frac{m_e}{m_i} N_0 \nue \vpe^2 \right> , \\
\label{gphisq_en_ev}
\pdiff{E_\phi}{t} & = &  \left< N_0 \phi \gradpar \vpe  - \nuin N_0 (\gradperp \phi)^2 - \mu_\phi \phi \gradperp^2 \varpi \right>, \\
\label{tesq_en_ev}
\pdiff{E_T}{t} & = &  \left<- \frac{3}{2} N_0 T_e {\mathbf v_E} \cdot \grad {\rm ln} T_{e0} - 1.71 N_0 T_e \gradpar \vpe - \kpe/T_{e0} (\gradpar T_e)^2  \right> \nonumber \\
& + & \left< - \frac{3 m_e}{m_i} \frac{N_0}{T_{e0}} \nue T_e^2  - \frac{3}{2} \mu_T \frac{N_0}{T_{e0}} (\gradperp T_e)^2 +  \frac{3}{2} \frac{N_0}{T_{e0}} T_e S_T  \right>,
\eeqar
where $E_v = \frac{1}{2} \left< N_0 \frac{m_e}{m_i} \vpe^2 \right>$, $E_\phi = \frac{1}{2} \left< N_0 (\gradperp \phi)^2 \right>$, and 
$E_T = \frac{3}{4} \left< P_0 (T_e/T_{e0})^2 \right>$. 

Note that there are a few simplifications made in these equations. One simplification is that the term
$\left< \mu_N \frac{T_{e0}}{N_0} N \gradperp^2 N \right>$ is written approximately as $- \left< \mu_N \frac{T_{e0}}{N_0} (\gradperp N)^2 \right>$ in Eq.~\ref{Nsq_en_ev}. 
The fact that $\frac{T_{e0}}{N_0} \approx 1$
makes this approximation acceptable. In fact, I don't use this approximation when calculating such quantities from the simulations, but I write it here as it illuminates the fact that
this energy term is negative. I use the same approximation with the $-\left< \frac{3}{2} \mu_T \frac{N_0}{T_{e0}} (\gradperp T)^2 \right>$ and $-\left< \kpe/T_{e0} (\gradpar T_e)^2 \right>$
terms, although the latter contains the fraction $\kpe/T_{e0}$, which is not necessarily close to being constant.

Moreover, notice that none of the advective nonlinear terms are present in these energy dynamics equations. 
The reason is that $\left< f \{ g,f \} \right> = 0$, which holds as long as all of the boundaries are periodic, have $f=0$ boundaries, or have $\grad g \cdot d\vec{S} = 0$ boundaries.
Now only Eq.~\ref{gphisq_en_ev} actually has this $\left< f \{ g,f \} \right> $ form for its nonlinearity
because all of the other energy equations contain equilibrium profile quantities in the volume average (e.g. $\left< \frac{T_{e0}}{N_0} N \{ \phi,N \} \right>$ in Eq.~\ref{Nsq_en_ev}).
Nevertheless, the equilibrium profile quantities come as $\frac{T_{e0}}{N_0} \approx 1$ for Eq.~\ref{Nsq_en_ev} and $\frac{N_0}{T_{e0}} \approx 1$ for Eq.~\ref{tesq_en_ev},
while there is a factor of the electron to ion mass ratio multiplied by the nonlinearity in Eq.~\ref{vesq_en_ev}. This means that all of the nonlinearities approximately vanish
in the energy equations. Furthermore, I have confirmed this by direct calculation of these terms. This is why I do not include the nonlinearities in Eqs.~\ref{Nsq_en_ev}-~\ref{tesq_en_ev}.

I note that I could have used a different expression for the energy in order to absolutely conserve the nonlinearities. For instance, I could have set $E_N = \frac{1}{2} \left< N^2 \right>$,
neglecting the factor $\frac{T_{e0}}{N_0}$. In fact, I did this in the Friedman et al. paper~\cite{friedman2012b}.
However, this expression would not be the physical energy, although it would have the convenient property of conserving the nonlinearities. 
Energy, after all, is a useful concept because it's a conserved quantity. Nevertheless, I have chosen to use the physical energy in this work as well as in another paper~\cite{friedman2013}
because the physical energy conserves the adiabatic response. I will show this below. Furthermore, the physical energy very nearly conserves the nonlinearities, 
so it's not a big problem to use the physical energy. The calculated error of neglecting the nonlinearities in the energy dynamics equations is only about $1 \%$.
Now, one may wonder why the physical energy doesn't absolutely conserve the advective nonlinearities. The answer
lies in the partial linearization of the simulation equations. The linearization neglects many nonlinear contributions that are needed for global energy conservation. Nevertheless, I find
that the spectral energy dynamics analysis in Sec.~\ref{s_spec_en_dyn} is simpler when I neglect most of the nonlinearities.

Now Eqs.~\ref{Nsq_en_ev}-~\ref{tesq_en_ev} are still not increadibly revealing because they contain nearly as many terms as the original simulated equations. However, I can break each
of these equations down in the following way:

\beq
\label{en_breakdown}
\pdiff{E_j}{t} = Q_j + C_j + D_j.
\eeq

The subscript $j$ represents the individual field: $(N, v, \phi, T)$. $Q_j$ represents energy injection from an equilibrium gradient. For example, $Q_N$ represents the energy injected
into $E_N$ (the density fluctuation potential energy) taken from the free energy of the equilibrium density gradient ($\grad_r N_0$). These terms are:

\beqar
\label{Q_N}
Q_N & = & \left< - T_{e0} N {\mathbf v_E} \cdot \grad {\rm ln} N_0 \right>, \\
\label{Q_v}
Q_v & = & 0, \\
\label{Q_phi}
Q_\phi & = & 0, \\
\label{Q_T}
Q_T & = & \left<- \frac{3}{2} N_0 T_e {\mathbf v_E} \cdot \grad {\rm ln} T_{e0} \right>.
\eeqar

Only the density and temperature fluctuations receive energy from the equilibrium density and temperature gradients, respectively. They do so by radial ${\bf E \times B}$ advection,
moving fluid or heat across the gradient where it can enhance or diminish the density and temperature fluctuations. I call the $Q_j$ terms energy injection terms, but they can in fact
dissipate fluctuation energy if the phase between the density (or temperature) and potential are stabilizing.

Next, the $C_j$ terms represent transfer channels. They are:

\beqar
\label{C_N}
C_N & = & \left< - T_{e0} N \gradpar \vpe \right>, \\
\label{C_v}
C_v & = & \left< - T_{e0} \vpe \gradpar N - 1.71 N_0 \vpe \gradpar T_e  + N_0 \vpe \gradpar \phi \right>, \\
\label{C_phi}
C_\phi & = & \left< N_0 \phi \gradpar \vpe  \right>, \\
\label{C_T}
C_T & = & \left< - 1.71 N_0 T_e \gradpar \vpe \right>.
\eeqar

Notice that $C_N + C_\phi + C_T = - C_v$ if the axial boundaries are periodic or zero value. 
Alternatively, $\sum_j C_j = 0$. So no energy is gained or lost in total. Energy does, however, transfer between the different fields:
$N, T_e, \phi \leftrightarrow \vpe$. All energy transfers through the parallel electron velocity. The density, temperature, and potential fluctuations all feed or draw energy from
the parallel electron velocity. The equations allow no state variable energy transfer. For instance, the density and potential fluctuations cannot transfer energy between each other directly.
Recall that this is the mechanism of the adiabatic response (see Sec.~\ref{s_lin_inst}). I commented on the conservation of energy of the adiabatic response above when discussing the use of 
the physical energy, and this is what I meant. Note that the ``energy-like'' expression used in one of our papers~\cite{friedman2012b} that absolutely conserved the advective nonlinearities
did not come close to conserving the adiabatic response energy. That's why in this work and in another paper~\cite{friedman2013}, I chose to use the physical energy.

I inject two minor points concerning the boundary conditions. The first is that the Neumann and sheath simulations don't exactly conserve the adiabatic response because of non-vanishing
contributions from the boundaries. Second, the sheath boundary conditions
allow energy transfer between the temperature and potential fluctuations without the adiabatic response. 
That transfer mechanism isn't represented in the $C_j$ expressions. I will calculate it in Chapter~\ref{c_nlin_nonper}.

Finally, the $D_j$ terms represent dissipative energy loss from the fluctuations. They are:

\beqar
\label{D_N}
D_N & = & \left< - \mu_N \frac{T_{e0}}{N_0} (\gradperp N)^2 + \frac{T_{e0}}{N_0} N S_N \right>, \\
\label{D_v}
D_v & = & \left<  - \frac{m_e}{m_i} N_0 \nue \vpe^2 \right>, \\
\label{D_phi}
D_\phi & = & \left<  - \nuin N_0 (\gradperp \phi)^2 - \mu_\phi \phi \gradperp^2 \varpi \right>, \\
\label{D_T}
D_T & = & \left<  - \kpe/T_{e0} (\gradpar T_e)^2 - \frac{3 m_e}{m_i} \frac{N_0}{T_{e0}} \nue T_e^2  \right> \nonumber \\
& + & \left< - \frac{3}{2} \mu_T \frac{N_0}{T_{e0}} (\gradperp T_e)^2 +  \frac{3}{2} \frac{N_0}{T_{e0}} T_e S_T \right>.
\eeqar

Most of these terms are clearly negative. However, the source terms do not have a clear sign and the $\left< - \mu_\phi \phi \gradperp^2 \varpi \right>$ viscous term in Eq.~\ref{D_phi} doesn't
have a clear sign either. Recall, though, that the sources essentially remove the flux-surface averaged component of the density and temperature fluctuations, indicating that they remove
the energy associated with these fluctuation components. Taking $S_N \approx - \left< N \right>_{fs}$ from Eqs.~\ref{Sn_eq} and~\ref{Sn_eq2}, then the source contribution to $D_N$ is
$- \left< \frac{T_{e0}}{N_0} \left< N \right>_{fs}^2 \right>$, which is negative. The viscous term in Eq.~\ref{D_phi} is less obviously negative, however, letting $\gradperp \rightarrow - k_\perp^2$
makes the viscous term approximately $- \left< \mu_\phi N_0 k_\perp^4 \phi^2 \right>$. So it's reasonable to conclude that all contributions in the $D_j$ expressions are absolutely negative.
My direct calculations have confirmed this.

\section{Spectral Energy Dynamics}
\label{s_spec_en_dyn}

While the total energy dynamics can reveal some important information such as the amount of energy entering the density fluctuations vs. the temperature fluctuations, the direction of
energy flow through the adiabatic response, and how much energy is dissipated by the various mechanisms, the total dynamics cannot show the mechanism of the nonlinear instability.
In fact, the total energy dynamics are rather useless in revealing any nonlinear physics. Spectral or mode-decomposed energy dynamics, on the other hand, provide much more information
regarding mode-specific processes like cascades and complex nonlinear processes.

When deriving mode-decomposed energy dynamics, one first has to choose a set of basis functions. As long as the functions form an independent complete basis, they are acceptable.
Fourier modes are a natural basis to use for any coordinate with periodic boundaries. Fourier modes are also orthogonal to one another, making them ideal. Linear eigenmodes provide
another good choice for a basis; however, they can be non-orthogonal in some systems, making them somewhat unweildy. I began this study using a linear eigenmode decomposition,
but I eventually gave up that path because the linear eigenmodes of this system are non-orthogonal. The results were complicated and didn't show anything more interesting than a simpler
Fourier decomposition would show. Hatch et al. dealt with this non-orthogonality problem with two distinct methods~\cite{hatch2011}. The first was to use a Gram-Schmidt orthogonalization
procedure starting with the most unstable linear eigenmode to make the modes orthogonal. The resulting modes other than the most unstable linear eigenmode, however, were no longer
the linear eigenmodes. The second method was to use a proper orthogonal decomposition (POD, which is essentially a kind of singular value decomposition) to create orthogonal modes that best
captured the dominant turbulent structures. Both of these methods are very interesting and quite useful for some systems, however, they are probably most useful for systems with
strong linear instabilities where most of the energy of the turbulence is contained in the fastest growing linear eigenmode or a dominant POD mode. That is not the case for the LAPD turbulence.

I found that the most useful basis to take is a partial Fourier basis. Namely, I simply use a Fourier decomposition in the azimuthal and axial directions.
For example, I decompose the density in the following way:

\beq
\label{density_decomp}
N(r,\theta,z,t) = \sum_{\vec{k}} n_{\vec{k}}(r,t) e^{i(m \theta + k_z z)}.
\eeq

Here, $k_z = \frac{2 \pi n}{L_\para \rho_s}$, where $n$ is the axial mode number and $m$ is the azimuthal mode number, and the $\vec{k}$ symbol is short for $(m,n)$.
The sum over $\vec{k}$ is in fact a double sum over $m$ and $n$. Furthermore, positive and negative
$m$ and $n$ are included in the sums to ensure reality of $N$ since $n_{-\vec{k}} = n_{\vec{k}}^*$.
Similar decompositions are used for $\vpe, \phi,$ and $T_e$. Note that the radial part of the basis function $n_{\vec{k}}(r,t)$ is time-dependent and shouldn't really be called a basis
function at all because of this. I cannot say anything about nonlinear processes involving different radial modes since I haven't decomposed the radial structures into a time-independent
basis. Essentially, I have limited the amount of information I can gain from this decomposition. I have intentionally done this to focus on a few particular results, which would be
more difficult to see if I used an additional radial decomposition.

To derive the spectral energy equations, I first substitute the Fourier decompositions into Eqs.~\ref{ni_eq}-~\ref{te_eq}. Using the density evolution equation as an example, I get:

\beqar
\label{density_eq_fourier}
& & \sum_{\vec{k}} \pdiff{n_{\vec{k}}}{t} e^{i(m \theta + k_z z)} = \nonumber \\
& & \sum_{\vec{k}} \left[ -\frac{i m}{r} \pdr N_0 \phi_{\vec{k}} - i k_z N_0 v_{\vec{k}} + \mu_N(\pdrr n_{\vec{k}} + \frac{1}{r} \pdr n_{\vec{k}} - \frac{m^2}{r^2} n_{\vec{k}}) \right] e^{i(m \theta + k_z z)} \nonumber \\
& & + \frac{1}{r} \sum_{\vec{k},\vec{k}'} (i m n_{\vec{k}} \pdr \phi_{\vec{k}'} - i m' \pdr n_{\vec{k}} \phi_{\vec{k}'}) e^{i (m + m') \theta + i (k_z + k'_z) z} + S_N.
\eeqar

Note the double sum for the nonlinearity. Continuing on with just the density equation for now, I proceed to get the energy equation by multiplying through by 
$ \frac{T_{e0}}{N_0} n_{\vec{k}''}^* e^{- i m'' \theta - i k''_z z}$ and integrating over space. The result is (with primes permutted):

\beqar
\label{density_evolution}
& &  \frac{1}{2} \left< \frac{T_{e0}}{N_0} \pdiff{|n_{\vec{k}}|^2}{t} \right> = \nonumber \\
& & \left< -\frac{T_{e0}}{N_0} \frac{i m}{r} \pdr N_0 \phi_{\vec{k}} n_{\vec{k}}^* - i k_z T_{e0} v_{\vec{k}} n_{\vec{k}}^* + \frac{T_{e0}}{N_0} \mu_N( \pdrr n_{\vec{k}} + \frac{1}{r} \pdr n_{\vec{k}} - \frac{m^2}{r^2} n_{\vec{k}}) n_{\vec{k}}^*  \right> \nonumber \\
& & + \left< \frac{T_{e0}}{r N_0} \sum_{\vec{k}'} \left( i m' n_{\vec{k}'} \pdr \phi_{\vec{k}-\vec{k}'} n_{\vec{k}}^*  - i (m - m') \pdr n_{\vec{k}'} \phi_{\vec{k}-\vec{k}'} n_{\vec{k}}^*  \right) \right> \nonumber \\
& & + \left< \frac{T_{e0}}{N_0} S_N n_{\vec{k}=0}^* \right>,
\eeqar
where the brackets now represent the reality operator and the radial integral $Re \left\{ \int r dr \right\}$ because I have performed the azimuthal and axial integrations and taken
the real part of the equation. Breaking this up into specific parts:

\beq
\label{Fourier_density_evolution}
\pdiff{E_N(\vec{k})}{t}  =  Q_N(\vec{k}) + C_N(\vec{k}) + D_N(\vec{k}) + \sum_{\vec{k}'} T_N(\vec{k},\vec{k}')
\eeq
with

\beqar
\label{ENk}
E_N(\vec{k}) & = & \frac{1}{2} \left< \frac{T_{e0}}{N_0} |n_{\vec{k}}|^2 \right> \\
\label{QNk}
Q_N(\vec{k}) & = & \left< -\frac{i m}{r} \frac{T_{e0}}{N_0} \pdr N_0 \phi_{\vec{k}} n_{\vec{k}}^* \right> \\
\label{CNk}
C_N(\vec{k}) & = & \left< - i k_z T_{e0} v_{\vec{k}} n_{\vec{k}}^* \right> \\
\label{DNk}
D_N(\vec{k}) & = & \left<  \frac{T_{e0}}{N_0} \mu_N( \pdrr n_{\vec{k}} + \frac{1}{r} \pdr n_{\vec{k}} - \frac{m^2}{r^2} n_{\vec{k}}) n_{\vec{k}}^*  + \frac{T_{e0}}{N_0} S_N n_{\vec{k}=0}^*  \right> \\
\label{TNk}
T_N(\vec{k},\vec{k}') & = & \left< \frac{T_{e0}}{r N_0} \left( i m' n_{\vec{k}'} \pdr \phi_{\vec{k}-\vec{k}'} n_{\vec{k}}^*  - i (m - m') \pdr n_{\vec{k}'} \phi_{\vec{k}-\vec{k}'} n_{\vec{k}}^* \right) \right>
\eeqar

The new piece not in the total energy dynamics in Sec.~\ref{s_tot_en_dyn}, $T_N(\vec{k},\vec{k}')$, comes from the advective nonlinearity. 
It couples different Fourier modes, meaning that it transfers energy between different $\vec{k}$ waves.
It is not conserved for individual $\vec{k}$ modes, but is conserved on the aggregate, meaning $\sum_{\vec{k}, \vec{k}'} T_N(\vec{k},\vec{k}') \simeq 0$.
Notice also that $Q_N(\vec{k})$ can be finite for $n=0$, but $C_N(\vec{k})$ is zero for $n=0$. In other words, flute modes may take energy from the equilibrium density gradient,
but they cannot access the adiabatic response. This precludes linear drift wave flute modes. But it does not preclude nonlinear drift wave flute modes because they can transfer their
energy to non-flute structures in order to access the adiabatic response.

For completeness, I write the rest of the spectral energy dynamics pieces here. The perpendicular kinetic energy dynamics pieces are:


\beqar
\label{Ephik}
E_\phi(\vec{k}) & = & \frac{1}{2} \left<  N_0 \left| \pdiff{\phi_{\vec{k}}}{r} \right|^2 + N_0 \frac{m^2}{r^2} |\phi_{\vec{k}}|^2  \right>\\
\label{Qphik}
Q_\phi(\vec{k}) & = & 0 \\
\label{Cphik}
C_\phi(\vec{k}) & = &  \left< i k_z N_0 v_{\vec{k}} \phi_{\vec{k}}^* \right>  \\
\label{Dphik}
D_\phi(\vec{k}) & = &  \left<  - \mu_\phi( \pdrr \varpi_{\vec{k}} + \frac{1}{r} \pdr \varpi_{\vec{k}} - \frac{m^2}{r^2} \varpi_{\vec{k}}) \phi_{\vec{k}}^* -  \nuin E_\phi(\vec{k})\right> \\
\label{Tphik}
T_\phi(\vec{k},\vec{k}') & = &  \left< - \frac{1}{r} \left( i m' \varpi_{\vec{k}'} \pdr \phi_{\vec{k}-\vec{k}'} \phi_{\vec{k}}^*  - i (m - m') \pdr \varpi_{\vec{k}'} \phi_{\vec{k}-\vec{k}'} \phi_{\vec{k}}^*     \right) \right> 
\eeqar
and for the electron temperature potential energy:

\beqar
\label{ETk}
E_T(\vec{k}) & = & \frac{3}{4} \left< \frac{N_0}{T_{e0}} |t_{\vec{k}}|^2  \right> \\
\label{QTk}
Q_T(\vec{k}) & = &  \left< - \frac{3}{2} \frac{N_0}{T_{e0}} \frac{i m}{r} \pdr T_{e0} \phi_{\vec{k}} t_{\vec{k}}^* \right>  \\
\label{CTk}
C_T(\vec{k}) & = &  \left<  - 1.71 i k_z N_0 v_{\vec{k}} t_{\vec{k}}^* \right>  \\
\label{DTk}
D_T(\vec{k}) & = &  \left< -\frac{\kpe}{T_{e0}} k_z^2 |t_{\vec{k}}|^2  - \frac{3 m_e}{m_i} \frac{N_0}{T_{e0}} \nue |t_{\vec{k}}|^2 \right> \nonumber \\
& + & \left< \frac{3}{2} \frac{N_0}{T_{e0}} \mu_T( \pdrr t_{\vec{k}} + \frac{1}{r} \pdr t_{\vec{k}} - \frac{m^2}{r^2} t_{\vec{k}}) t_{\vec{k}}^*  + \frac{3}{2} \frac{N_0}{T_{e0}} S_T t_{\vec{k}=0}^*  \right>  \\
\label{TTk}
T_T(\vec{k},\vec{k}') & = &  \left< \frac{3}{2 r} \frac{N_0}{T_{e0}} \left( i m' t_{\vec{k}'} \pdr \phi_{\vec{k}-\vec{k}'} t_{\vec{k}}^*  - i (m - m') \pdr t_{\vec{k}'} \phi_{\vec{k}-\vec{k}'} t_{\vec{k}}^*   \right) \right>
\eeqar
and for the parallel kinetic energy:

\beqar
\label{Evk}
E_v(\vec{k}) & = & \frac{1}{2} \fmei \left< N_0 |v_{\vec{k}}|^2 \right> \\
\label{Qvk}
Q_v(\vec{k}) & = & 0 \\
\label{Cvk}
C_v(\vec{k}) & = & \left< - i k_z N_0 n_{\vec{k}} v_{\vec{k}}^* + i k_z N_0 \phi_{\vec{k}} v_{\vec{k}}^* - 1.71 i k_z T_{e0} t_{\vec{k}} v_{\vec{k}}^*  \right> \\
\label{Dvk}
D_v(\vec{k}) & = & \left< - \nue \fmei N_0 |v_{\vec{k}}|^2   \right> \\
\label{Tvk}
T_v(\vec{k},\vec{k}') & = & \left< \fmei \frac{N_0}{r} \left( i m' v_{\vec{k}'} \pdr \phi_{\vec{k}-\vec{k}'} v_{\vec{k}}^*  - i (m - m') \pdr v_{\vec{k}'} \phi_{\vec{k}-\vec{k}'} v_{\vec{k}}^*   \right) \right>.
\eeqar


