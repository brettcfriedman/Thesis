\chapter{Energy Dynamics Formalism}
\label{c_en_formalism}

In the last section of the previous chapter, I analyzed the experimental and simulated turbulence using simple and common statistical methods. Never did I assume any kind of model for the turbulence,
nor did I take full advantage of the wealth of spatial information provided by the simulations. In the next few chapters, I do use the simulated physics model along with the turbulent spatial
structures to analyze the nature of the turbulence from an energy dynamics perspective. The energy dynamics provide direct information about energy injection into the turbulence from the equilibrium
gradients, energy transfer among different fields and between different normal modes, and turbulent energy dissipation. This information reveals the mysterious mechanism that
drives the $n=0$ fluctuations so strongly in the simulations (see Fig.~\ref{n_statistics} f)). The mechanism is a nonlinear instability. I will provide evidence
for this in the next chapter, but in this chapter, I derive the dynamical energy equations and explain what they mean.

\section{Total Energy and Dynamics}
\label{s_tot_en_dyn}

First, I consider the total, volume-averaged energy and energy dynamics. The total volume-averaged energy of the fluctuations (in normalized units) is:
\beq
\label{tot_energy}
E = \frac{1}{2} \int_V  \left[ P_0 \left((N/N_0)^2 + \frac{3}{2} (T_e/T_{e0})^2 \right) + N_0 \left( \frac{m_e}{m_i} \vpe^2 + (\gradperp \phi)^2 \right) \right] dV,
\eeq
where $P_0 = N_0 T_{e0}$ is the equilibrium pressure.
The $\frac{1}{2} P_0 (N/N_0)^2$ term is the potential energy due to density fluctuations, $\frac{3}{4} P_0 (T_e/T_{e0})^2$ is the electron temperature fluctuation potential energy,
$\frac{1}{2} N_0 \frac{m_e}{m_i} \vpe^2$ is the parallel electron kinetic energy, and $\frac{1}{2} N_0 (\gradperp \phi)^2$ is the ${\bf E \times B}$ perpendicular kinetic energy.
The energy contained in the electric field is smaller than the perpendicular kinetic energy by a factor of $(v_A/c)^2$ and is therefore neglected.

I obtain the dynamical energy evolution $\partial E/\partial t$ using Eqs.~\ref{ni_eq}-~\ref{te_eq} in the following way: 
first, I take Eq.~\ref{ni_eq}, multiply both sides by $\frac{T_{e0}}{N_0} N$, and integrate it over the volume.
The result is:

\beq
\label{Nsq_en_ev}
\pdiff{E_N}{t} = \left< - T_{e0} N {\mathbf v_E} \cdot \grad {\rm ln} N_0 - T_{e0} N \gradpar \vpe - \mu_N \frac{T_{e0}}{N_0} (\gradperp N)^2 + \frac{T_{e0}}{N_0} N S_N \right>,
\eeq
where $E_N = \frac{1}{2} \left< P_0 (N/N_0)^2 \right>$ with $\left< \right>$ shorthand for the volume integral $\int_V dV$. 
Next, I multiply Eq.~\ref{ve_eq} by $N_0 \frac{m_e}{m_i} \vpe$, Eq.~\ref{rho_eq} by $- \phi$, and Eq.~\ref{te_eq}
by $\frac{3}{2} \frac{N_0}{T_{e0}} T_e$ and volume integrate, giving:

\beqar
\label{vesq_en_ev}
\pdiff{E_v}{t} & = &  \left< - T_{e0} \vpe \gradpar N - 1.71 N_0 \vpe \gradpar T_e  + N_0 \vpe \gradpar \phi - \frac{m_e}{m_i} N_0 \nue \vpe^2 \right> , \\
\label{gphisq_en_ev}
\pdiff{E_\phi}{t} & = &  \left< N_0 \phi \gradpar \vpe  - \nuin N_0 (\gradperp \phi)^2 - \mu_\phi \phi \gradperp^2 \varpi \right>, \\
\label{tesq_en_ev}
\pdiff{E_T}{t} & = &  \left<- \frac{3}{2} N_0 T_e {\mathbf v_E} \cdot \grad {\rm ln} T_{e0} - 1.71 N_0 T_e \gradpar \vpe - \kpe/T_{e0} (\gradpar T_e)^2  \right> \nonumber \\
& + & \left< - \frac{3 m_e}{m_i} \frac{N_0}{T_{e0}} \nue T_e^2  - \frac{3}{2} \mu_T \frac{N_0}{T_{e0}} (\gradperp T_e)^2 +  \frac{3}{2} \frac{N_0}{T_{e0}} T_e S_T  \right>,
\eeqar
where $E_v = \frac{1}{2} \left< N_0 \frac{m_e}{m_i} \vpe^2 \right>$, $E_\phi = \frac{1}{2} \left< N_0 (\gradperp \phi)^2 \right>$, and 
$E_T = \frac{3}{4} \left< P_0 (T_e/T_{e0})^2 \right>$. 

Note that there are a few simplifications made in these equations. One simplification is that the term
$\left< \mu_N \frac{T_{e0}}{N_0} N \gradperp^2 N \right>$ is written approximately as $- \left< \mu_N \frac{T_{e0}}{N_0} (\gradperp N)^2 \right>$ in Eq.~\ref{Nsq_en_ev}. 
The fact that $\frac{T_{e0}}{N_0} \approx 1$
makes this approximation acceptable. In fact, I don't use this approximation when calculating such quantities from the simulations, but I write it here as it illuminates the fact that
this energy term is negative. I use the same approximation with the $-\left< \frac{3}{2} \mu_T \frac{N_0}{T_{e0}} (\gradperp T)^2 \right>$ and $-\left< \kpe/T_{e0} (\gradpar T_e)^2 \right>$
terms, although the latter contains the fraction $\kpe/T_{e0}$, which is not necessarily close to being constant.

Moreover, notice that none of the advective nonlinear terms are present in these energy dynamics equations. 
The reason is that $\left< f \{ g,f \} \right> = 0$, which holds as long as $f$ and $g$ have periodic boundary conditions, or if any boundaries are not periodic, 
then the boundaries must satisfy $f=0$ and $\grad g \cdot d\vec{S} = 0$.
Now only Eq.~\ref{gphisq_en_ev} actually has this $\left< f \{ g,f \} \right> $ form for its nonlinearity
because all of the other energy equations contain equilibrium profile quantities in the volume average (e.g. $\left< \frac{T_{e0}}{N_0} N \{ \phi,N \} \right>$ in Eq.~\ref{Nsq_en_ev}).
Nevertheless, the equilibrium profile quantities come as $\frac{T_{e0}}{N_0} \approx 1$ for Eq.~\ref{Nsq_en_ev} and $\frac{N_0}{T_{e0}} \approx 1$ for Eq.~\ref{tesq_en_ev},
while there is a factor of the electron to ion mass ratio multiplied by the nonlinearity in Eq.~\ref{vesq_en_ev}. This means that all of the nonlinearities approximately vanish
in the energy equations. I have confirmed this by direct calculation of these terms. This is why I do not include the nonlinearities in Eqs.~\ref{Nsq_en_ev}-~\ref{tesq_en_ev}.

I note that I could have used a different expression for the energy in order to absolutely conserve the nonlinearities. For instance, I could have set $E_N = \frac{1}{2} \left< N^2 \right>$,
neglecting the factor $\frac{T_{e0}}{N_0}$. In fact, I did this in one paper~\cite{friedman2012b}.
However, even though such an expression has the nice property of conserving the nonlinearities, it does not conserve the adiabatic response (I show below how the energy I use does conserve
the adiabatic response).
Energy is a useful concept because of its conservation properties, but unfortunately, in this case, I have to choose which property to conserve. I choose here to use the physical energy 
-- Eq.~\ref{tot_energy} -- that conserves the adiabatic response for a few reasons (I also used the physical energy in another paper~\cite{friedman2013}).
First, the adiabatic-conserving energy is the same thing as the physical energy -- meaning, it is the energy one would write down without knowledge of the equations. 
Second, the physical energy generally conserves the nonlinearities more than the nonlinearity-conserving energy conserves the adiabatic response. I have tested this using my simulations.
And third, when the adiabatic response is not conserved, the energy dynamics appear to deposit energy directly into the $\vpe$ fluctuations, which is not standard.

Now, one may wonder why the physical energy doesn't absolutely conserve the advective nonlinearities when many other papers use a physical energy that does. For one, those other papers
generally use local rather than global models. Additionally, I have partially linearized the simulation equations, eliminating certain conservation properties.
Still though, this partial linearization makes the energy dynamics analysis simpler, and therefore more useful, although some may disagree.

Now Eqs.~\ref{Nsq_en_ev}-~\ref{tesq_en_ev} are still not incredibly revealing because they contain nearly as many terms as the original simulated equations. However, I can break each
of these equations down in the following way:
\beq
\label{en_breakdown}
\pdiff{E_j}{t} = Q_j + C_j + D_j.
\eeq
The subscript $j$ represents the individual field: $(N, v, \phi, T)$. $Q_j$ represents energy injection from an equilibrium gradient. For example, $Q_N$ represents the energy injected
into $E_N$ (the density fluctuation potential energy) taken from the free energy of the equilibrium density gradient ($\grad_r N_0$). These terms are:
\beqar
\label{Q_N}
Q_N & = & \left< - T_{e0} N {\mathbf v_E} \cdot \grad {\rm ln} N_0 \right>, \\
\label{Q_v}
Q_v & = & 0, \\
\label{Q_phi}
Q_\phi & = & 0, \\
\label{Q_T}
Q_T & = & \left<- \frac{3}{2} N_0 T_e {\mathbf v_E} \cdot \grad {\rm ln} T_{e0} \right>.
\eeqar
Only the density and temperature fluctuations receive energy from the equilibrium density and temperature gradients, respectively. They do so by radial ${\bf E \times B}$ advection,
moving fluid or heat across the gradient where it can enhance or diminish the density and temperature fluctuations. I call the $Q_j$ terms energy injection terms, but they can in fact
dissipate fluctuation energy if the phase between the density (or temperature) and potential are stabilizing.

Next, the $C_j$ terms, which constitute the adiabatic response, represent field transfer channels. They are:
\beqar
\label{C_N}
C_N & = & \left< - T_{e0} N \gradpar \vpe \right>, \\
\label{C_v}
C_v & = & \left< - T_{e0} \vpe \gradpar N - 1.71 N_0 \vpe \gradpar T_e  + N_0 \vpe \gradpar \phi \right>, \\
\label{C_phi}
C_\phi & = & \left< N_0 \phi \gradpar \vpe  \right>, \\
\label{C_T}
C_T & = & \left< - 1.71 N_0 T_e \gradpar \vpe \right>.
\eeqar
Notice that $C_N + C_\phi + C_T = - C_v$ if the axial boundaries are periodic or zero value. 
Alternatively, $\sum_j C_j = 0$. This is what I mean by conservation of the adiabatic response. No energy is gained or lost in total from these terms when taken together. 
Energy does, however, transfer between the different fields: $N, T_e, \phi \leftrightarrow \vpe$. All energy transfers through the parallel electron velocity. 
The density, temperature, and potential fluctuations all feed or draw energy from the parallel electron velocity. 
This means that the density and potential fluctuations, for instance, cannot transfer energy between each other directly.
There are two minor points regarding the affect of the boundary conditions on the adiabatic response. 
First, the Neumann and sheath simulations don't exactly conserve the adiabatic response because of non-vanishing contributions from the boundaries. Second, the sheath boundary conditions
allow energy transfer between the temperature and potential fluctuations that is completely independent of the adiabatic response. In fact, this is what allows for the CWM (see Sec.~\ref{ss_cwm}).
The sheath energy transfer mechanism isn't represented by the $C_j$ expressions. I don't account for that transfer mechanism in these energy dynamics equations. I leave it to future work.

Finally, the $D_j$ terms represent dissipative energy loss from the fluctuations. They are:
\beqar
\label{D_N}
D_N & = & \left< - \mu_N \frac{T_{e0}}{N_0} (\gradperp N)^2 + \frac{T_{e0}}{N_0} N S_N \right>, \\
\label{D_v}
D_v & = & \left<  - \frac{m_e}{m_i} N_0 \nue \vpe^2 \right>, \\
\label{D_phi}
D_\phi & = & \left<  - \nuin N_0 (\gradperp \phi)^2 - \mu_\phi \phi \gradperp^2 \varpi \right>, \\
\label{D_T}
D_T & = & \left<  - \kpe/T_{e0} (\gradpar T_e)^2 - \frac{3 m_e}{m_i} \frac{N_0}{T_{e0}} \nue T_e^2  \right> \nonumber \\
& + & \left< - \frac{3}{2} \mu_T \frac{N_0}{T_{e0}} (\gradperp T_e)^2 +  \frac{3}{2} \frac{N_0}{T_{e0}} T_e S_T \right>.
\eeqar
Most of these terms have forms that illustrate that their negative definiteness. 
However, the source terms do not have a clear sign and the $\left< - \mu_\phi \phi \gradperp^2 \varpi \right>$ viscous term in Eq.~\ref{D_phi} doesn't
have a clear sign either. Recall, though, that the sources essentially remove the flux-surface averaged component of the density and temperature fluctuations, indicating that they remove
the energy associated with these fluctuation components. Taking $S_N \approx - \left< N \right>_{fs}$ from Eqs.~\ref{Sn_eq} and~\ref{Sn_eq2}, then the source contribution to $D_N$ is
$- \left< \frac{T_{e0}}{N_0} \left< N \right>_{fs}^2 \right>$, which is negative. The viscous term in Eq.~\ref{D_phi} is less obviously negative, however, letting $\gradperp \rightarrow - k_\perp^2$
makes the viscous term approximately $- \left< \mu_\phi N_0 k_\perp^4 \phi^2 \right>$. So it's reasonable to conclude that all contributions in the $D_j$ expressions are absolutely negative.
My direct calculations have confirmed this.

\section{Spectral Energy Dynamics}
\label{s_spec_en_dyn}

While the total energy dynamics can reveal some important information such as the amount of energy entering the density fluctuations vs. the temperature fluctuations, the direction of
energy flow through the adiabatic response, and how much energy is dissipated by the various mechanisms, the total dynamics cannot show the mechanism of the nonlinear instability.
In fact, the total energy dynamics are rather useless in revealing any nonlinear physics. Spectral or mode-decomposed energy dynamics, on the other hand, provide much more information
regarding mode-specific processes like cascades and complex nonlinear processes. 

When deriving mode-decomposed energy dynamics, one first has to choose a set of basis functions (modes) on which to decompose the fluctuations. This is important because a good choice
of basis functions can immediately reveal important dynamical information, while a poor choice can lead to a lot of wasted time and a muddled picture.
Generally, basis functions are time-independent spatial structures that are linearly
independent and span the whole computational space $\Omega$. For example, a particular set of basis functions $\psi_i(\vec{r})$ can linearly sum to represent any 3D function on $\Omega$:
\beq
\label{func_decomp}
f(\vec{r}) = \sum_i a_i \psi_i(\vec{r}).
\eeq
In dynamical systems, the system is represented by a time-dependent 3D function, causing the amplitudes $a_i$ to vary with time:
\beq
\label{func_decomp_time}
f(\vec{r},t) = \sum_i a_i(t) \psi_i(\vec{r}).
\eeq
Fourier modes or linear eigenvectors are common examples of basis functions. However, Fourier modes are not always a useful or a natural basis, 
and linear eigenmodes can be unwieldy when they are nonorthogonal to each other, which is the case for my dynamical system. In the next section, I discuss an alternative basis, 
namely that obtained by Proper Orthogonal Decomposition, but for now, I describe a basis upon which I base most of my results. That basis is a partial Fourier basis, which I have found useful 
in analyzing the simulations and uncovering interesting physics.

What I mean by partial Fourier basis is that I decompose the azimuthal and axial directions in Fourier series, leaving the radial direction undecomposed.
For example, I decompose the density in the following way:
\beq
\label{density_decomp}
N(r,\theta,z,t) = \sum_{\vec{k}} n_{\vec{k}}(r,t) e^{i(m \theta + k_z z)}.
\eeq
Here, $k_z = \frac{2 \pi n}{L_\para \rho_s}$, where $n$ is the axial mode number and $m$ is the azimuthal mode number, and the $\vec{k}$ symbol is short for $(m,n)$.
The sum over $\vec{k}$ is in fact a double sum over $m$ and $n$. Furthermore, positive and negative
$m$ and $n$ are included in the sums to ensure reality of $N$ since $n_{-\vec{k}} = n_{\vec{k}}^*$.
Similar decompositions are used for $\vpe, \phi,$ and $T_e$. Note that the radial part of the basis function $n_{\vec{k}}(r,t)$ isn't really a basis function in the general sense. First,
it is time-dependent. Second, it doesn't span the radial domain. In fact, at a particular time, it only describes one very particular 1D (complex) function. Nevertheless, by not using a 
radial decomposition, I greatly reduce the number of modes of the problem, allowing me to focus on certain processes of interest.

Now, to derive the spectral energy equations, I first substitute the basis decompositions -- Eq.~\ref{density_decomp} and those corresponding to the other fields -- into Eqs.~\ref{ni_eq}-~\ref{te_eq}. 
Using the density evolution equation as an example, I get:
\beqar
\label{density_eq_fourier}
& & \sum_{\vec{k}} \pdiff{n_{\vec{k}}}{t} e^{i(m \theta + k_z z)} = \nonumber \\
& & \sum_{\vec{k}} \left[ -\frac{i m}{r} \pdr N_0 \phi_{\vec{k}} - i k_z N_0 v_{\vec{k}} + \mu_N(\pdrr n_{\vec{k}} + \frac{1}{r} \pdr n_{\vec{k}} - \frac{m^2}{r^2} n_{\vec{k}}) \right] e^{i(m \theta + k_z z)} \nonumber \\
& & + \frac{1}{r} \sum_{\vec{k},\vec{k}'} (i m n_{\vec{k}} \pdr \phi_{\vec{k}'} - i m' \pdr n_{\vec{k}} \phi_{\vec{k}'}) e^{i (m + m') \theta + i (k_z + k'_z) z} + S_N.
\eeqar
Note the double sum for the nonlinearity. Continuing on with just the density equation for now, I proceed to get the energy equation by multiplying through by 
$ \frac{T_{e0}}{N_0} n_{\vec{k}''}^* e^{- i m'' \theta - i k''_z z}$ and integrating over space. The result is (with primes permuted):
\beqar
\label{density_evolution}
& &  \frac{1}{2} \left< \frac{T_{e0}}{N_0} \pdiff{|n_{\vec{k}}|^2}{t} \right> = \nonumber \\
& & \left< -\frac{T_{e0}}{N_0} \frac{i m}{r} \pdr N_0 \phi_{\vec{k}} n_{\vec{k}}^* - i k_z T_{e0} v_{\vec{k}} n_{\vec{k}}^* + \frac{T_{e0}}{N_0} \mu_N( \pdrr n_{\vec{k}} + \frac{1}{r} \pdr n_{\vec{k}} - \frac{m^2}{r^2} n_{\vec{k}}) n_{\vec{k}}^*  \right> \nonumber \\
& & + \left< \frac{T_{e0}}{r N_0} \sum_{\vec{k}'} \left( i m' n_{\vec{k}'} \pdr \phi_{\vec{k}-\vec{k}'} n_{\vec{k}}^*  - i (m - m') \pdr n_{\vec{k}'} \phi_{\vec{k}-\vec{k}'} n_{\vec{k}}^*  \right) \right> \nonumber \\
& & + \left< \frac{T_{e0}}{N_0} S_N n_{\vec{k}}^* \delta_{\vec{k},0}\right>,
\eeqar
where the brackets now represent the reality operator and the radial integral $Re \left\{ \int r dr \right\}$ because I have performed the azimuthal and axial integrations and taken
the real part of the equation. Breaking this up into specific parts:
\beq
\label{Fourier_density_evolution}
\pdiff{E_N(\vec{k})}{t}  =  Q_N(\vec{k}) + C_N(\vec{k}) + D_N(\vec{k}) + \sum_{\vec{k}'} T_N(\vec{k},\vec{k}')
\eeq
with
\beqar
\label{ENk}
E_N(\vec{k}) & = & \frac{1}{2} \left< \frac{T_{e0}}{N_0} |n_{\vec{k}}|^2 \right> \\
\label{QNk}
Q_N(\vec{k}) & = & \left< -\frac{i m}{r} \frac{T_{e0}}{N_0} \pdr N_0 \phi_{\vec{k}} n_{\vec{k}}^* \right> \\
\label{CNk}
C_N(\vec{k}) & = & \left< - i k_z T_{e0} v_{\vec{k}} n_{\vec{k}}^* \right> \\
\label{DNk}
D_N(\vec{k}) & = & \left<  \frac{T_{e0}}{N_0} \mu_N( \pdrr n_{\vec{k}} + \frac{1}{r} \pdr n_{\vec{k}} - \frac{m^2}{r^2} n_{\vec{k}}) n_{\vec{k}}^*  + \frac{T_{e0}}{N_0} S_N n_{\vec{k}}^* \delta_{\vec{k},0} \right> \\
\label{TNk}
T_N(\vec{k},\vec{k}') & = & \left< \frac{T_{e0}}{r N_0} \left( i m' n_{\vec{k}'} \pdr \phi_{\vec{k}-\vec{k}'} n_{\vec{k}}^*  - i (m - m') \pdr n_{\vec{k}'} \phi_{\vec{k}-\vec{k}'} n_{\vec{k}}^* \right) \right>
\eeqar
The new piece not in the total energy dynamics in Sec.~\ref{s_tot_en_dyn} -- $T_N(\vec{k},\vec{k}')$ -- comes from the advective nonlinearity. 
It couples different Fourier modes, meaning it transfers energy between different $\vec{k}$ waves.
It is not conserved for individual $\vec{k}$ modes, but is conserved on the aggregate: $\sum_{\vec{k}, \vec{k}'} T_N(\vec{k},\vec{k}') \simeq 0$.
Notice also that $Q_N(\vec{k})$ can be finite for $n=0$, but $C_N(\vec{k})$ is zero for $n=0$. Thus, flute modes may take energy from the equilibrium density gradient,
but they cannot access the adiabatic response. This eliminates linear drift wave flute modes, but does not preclude nonlinear drift wave flute modes that transfer their
energy to non-flute structures in order to access the adiabatic response.

For completeness, I write the rest of the spectral energy dynamics pieces here. The perpendicular kinetic energy dynamics pieces are:
\beqar
\label{Ephik}
E_\phi(\vec{k}) & = & \frac{1}{2} \left<  N_0 \left| \pdiff{\phi_{\vec{k}}}{r} \right|^2 + N_0 \frac{m^2}{r^2} |\phi_{\vec{k}}|^2  \right>\\
\label{Qphik}
Q_\phi(\vec{k}) & = & 0 \\
\label{Cphik}
C_\phi(\vec{k}) & = &  \left< i k_z N_0 v_{\vec{k}} \phi_{\vec{k}}^* \right>  \\
\label{Dphik}
D_\phi(\vec{k}) & = &  \left<  - \mu_\phi( \pdrr \varpi_{\vec{k}} + \frac{1}{r} \pdr \varpi_{\vec{k}} - \frac{m^2}{r^2} \varpi_{\vec{k}}) \phi_{\vec{k}}^* -  \nuin E_\phi(\vec{k})\right> \\
\label{Tphik}
T_\phi(\vec{k},\vec{k}') & = &  \left< - \frac{1}{r} \left( i m' \varpi_{\vec{k}'} \pdr \phi_{\vec{k}-\vec{k}'} \phi_{\vec{k}}^*  - i (m - m') \pdr \varpi_{\vec{k}'} \phi_{\vec{k}-\vec{k}'} \phi_{\vec{k}}^*     \right) \right> 
\eeqar
and for the electron temperature potential energy:
\beqar
\label{ETk}
E_T(\vec{k}) & = & \frac{3}{4} \left< \frac{N_0}{T_{e0}} |t_{\vec{k}}|^2  \right> \\
\label{QTk}
Q_T(\vec{k}) & = &  \left< - \frac{3}{2} \frac{N_0}{T_{e0}} \frac{i m}{r} \pdr T_{e0} \phi_{\vec{k}} t_{\vec{k}}^* \right>  \\
\label{CTk}
C_T(\vec{k}) & = &  \left<  - 1.71 i k_z N_0 v_{\vec{k}} t_{\vec{k}}^* \right>  \\
\label{DTk}
D_T(\vec{k}) & = &  \left< -\frac{\kpe}{T_{e0}} k_z^2 |t_{\vec{k}}|^2  - \frac{3 m_e}{m_i} \frac{N_0}{T_{e0}} \nue |t_{\vec{k}}|^2 \right> \nonumber \\
& + & \left< \frac{3}{2} \frac{N_0}{T_{e0}} \mu_T( \pdrr t_{\vec{k}} + \frac{1}{r} \pdr t_{\vec{k}} - \frac{m^2}{r^2} t_{\vec{k}}) t_{\vec{k}}^*  + \frac{3}{2} \frac{N_0}{T_{e0}} S_T t_{\vec{k}}^* \delta_{\vec{k},0} \right>  \\
\label{TTk}
T_T(\vec{k},\vec{k}') & = &  \left< \frac{3}{2 r} \frac{N_0}{T_{e0}} \left( i m' t_{\vec{k}'} \pdr \phi_{\vec{k}-\vec{k}'} t_{\vec{k}}^*  - i (m - m') \pdr t_{\vec{k}'} \phi_{\vec{k}-\vec{k}'} t_{\vec{k}}^*   \right) \right>
\eeqar
and for the parallel kinetic energy:
\beqar
\label{Evk}
E_v(\vec{k}) & = & \frac{1}{2} \fmei \left< N_0 |v_{\vec{k}}|^2 \right> \\
\label{Qvk}
Q_v(\vec{k}) & = & 0 \\
\label{Cvk}
C_v(\vec{k}) & = & \left< - i k_z N_0 n_{\vec{k}} v_{\vec{k}}^* + i k_z N_0 \phi_{\vec{k}} v_{\vec{k}}^* - 1.71 i k_z T_{e0} t_{\vec{k}} v_{\vec{k}}^*  \right> \\
\label{Dvk}
D_v(\vec{k}) & = & \left< - \nue \fmei N_0 |v_{\vec{k}}|^2   \right> \\
\label{Tvk}
T_v(\vec{k},\vec{k}') & = & \left< \fmei \frac{N_0}{r} \left( i m' v_{\vec{k}'} \pdr \phi_{\vec{k}-\vec{k}'} v_{\vec{k}}^*  - i (m - m') \pdr v_{\vec{k}'} \phi_{\vec{k}-\vec{k}'} v_{\vec{k}}^*   \right) \right>.
\eeqar


\section{Proper Orthogonal Decomposition}
\label{s_pod}

\subsection{Decomposition}
\label{ss_pod_decomp}

As I alluded to above, there are many choices by which one can mode-decompose a turbulent system. Full Fourier decompositions and linear eigenmode decompositions are common. I choose
not to use a radial Fourier decomposition for a couple of reasons. First, the radial Fourier modes are poor representatives of the turbulent structures. 
That is, too many radial Fourier modes have large coefficients upon decomposition, which makes simple modeling difficult. Second, the equilibrium profiles have radial dependence,
and some of the differential operators contain factors of the radius $r$. This means that the integration over the volume which eliminates the Fourier exponentials 
-- like the step between Eq.~\ref{density_eq_fourier} and Eq.~\ref{density_evolution} -- does not work for the radial coordinate. This is more of an aesthetic consideration than a mathematical one;
nevertheless, I prefer to avoid it.

The linear eigenmode decomposition is an attractive one because the dynamical energy expressions can be written elegantly (see Sec.~\ref{ss_nl_sat_levels}), and it seems that the fastest growing
linear eigenmodes should make up most of the turbulent amplitude. 
However, there is the practical difficulty in doing an eigenmode decomposition in that one has to somehow find all of the linear eigenvectors, 
which cannot be done with an initial value code like BOUT++. One must write or use an eigensystem code to do this. 
On a more fundamental level, however, dynamical systems with non-normal linear operators -- gradient-driven systems --
have nonorthogonal linear eigenvectors. Using a nonorthogonal basis decomposition can be too unwieldy for a decomposition analysis because the total energy 
contains contributions from each individual eigenmode plus contributions from cross terms. I showed a simple example of this in Sec.~\ref{s_nonormality}. 
The fact that the cross terms can have negative energies is undesirable and unmanageable. Using left and right eigenvectors to introduce some kind of orthogonality condition can
partly simplify matters, but energies and the dynamical terms still contain cross terms, leaving still overly complicated results~\cite{kim2010}.
I, in fact, began this line of research using an eigenmode decomposition, but I eventually
gave up that path because the results were too complicated and the interesting physics didn't depend on that particular decomposition.

Hatch et al. dealt with the nonorthogonal eigenvector problem using two distinct methods~\cite{hatch2011}. The first was to use a Gram-Schmidt orthogonalization
procedure, retaining the most unstable linear eigenmode and orthogonalizing the others from this. The resulting orthogonal modes other than the most unstable linear eigenmode, however, are not
linear eigenmodes after the procedure. They more or less form an arbitrary orthogonal basis, leaving this method with limited applicability. 
The second method they used was Proper Orthogonal Decomposition (POD, aka Principle Component Analysis) to create orthogonal modes that best
captured the dominant turbulent structures. The POD has properties that make it the most desirable decomposition for my dynamical system -- other than the one in the previous section.

POD is a procedure for extracting an orthogonal basis from an ensemble of space-time signals. Its power lies in its generality, its linearity, and its creation of an ``optimal'' basis.
A nice review of the properties of POD is given by Berkooz et al.~\cite{berkooz1993}. A less descriptive and less rigorous description of POD is given in Futatani et al.~\cite{futatani2009},
and I will follow their treatment to show how to construct the POD and to present some of its properties. Simply, the POD is a singular value decomposition (SVD) of the data given by
\beq
\label{svd}
A(\vec{r}_i,t_j) = \sum_{q=1}^{N_{\rm{POD}}} \sigma_q u_q(\vec{r}_i) w_q(t_j)
\eeq
where $A(\vec{r},t)$ is the data.  In my case, four independent variables
$(N, \phi, \vpe, T_e)$ comprise the data. These must be appended together to get the full matrix $A$. Furthermore, $N_{\rm{POD}} = \rm{min}[4 \times N_r \times N_\theta \times N_z,N_t]$. 
In other words, $N_{\rm{POD}}$ is the lesser
of 4 times the number of total grid points -- the degrees of freedom -- and the number of time points. For me, $N_{\rm{POD}} = N_t$ because I choose to retain more spatial data than time data.
Linear eigenmodes and full Fourier modes, on the other hand, always number $4 \times N_r \times N_\theta \times N_z$. This means that the spatial POD functions, $u_q(\vec{r})$,
do not span the computational domain -- any arbitrary function on the computational domain cannot be represented by a linear combination of the $u_q$ POD functions. 
They do, however, span a subspace of the domain and every data signal that is used to derive the $u_q$ POD functions can be represented by a linear combination of them.

Before I continue, I note a couple of practical considerations. First, I find it useful at times to first Fourier decompose the data in the azimuthal and axial dimensions before performing the POD.
In that case, the data for each Fourier pair $\vec{k}=(m,n)$ is only a function of the radial coordinate $r$ and time: $A_{\vec{k}}(r,t)$. The spatial POD functions $u_q$ are then 1D (complex) 
functions of radius. Second, if I do not Fourier decompose the data, I must make $A$ a 2D matrix in order to take the SVD computationally. To do this, I simply unravel or 
collapse all of the spatial dimensions into a 1D vector. Then a single column of $A$ is the unraveled spatial data at one particular instant in time, and the time varies from column to column.
Furthermore, the non-Fourier spatial and temporal POD functions $u_q$ and $w_q$ are real, not complex. Third, whether or not I use a Fourier decomposition, $A$ must be a 2D matrix and $u_q$
must be a 1D vector, so performing the POD requires appending the four fields $(N, \phi, \vpe, T_e)$ into a single vector.

Continuing on, the spatial POD $u_q$ modes and the temporal POD $w_q$ modes satisfy the following orthonormality conditions:
\beq
\label{pod_orthonormality}
\sum_{i=1}^{N_{\rm{POD}}} u_q(\vec{r}_i) u^*_l(\vec{r}_i) = \sum_{j=1}^{N_{\rm{POD}}} w_q(t_j) w^*_l(t_j) = \delta_{q l}.
\eeq
The positive real numbers $\sigma_q$ are the singular values, and they are sorted in descending order, i.e., $\sigma_1 \ge \sigma_2 \ge \sigma_3 \cdots$. Then for $1 \le h \le N_{\rm{POD}}$, I
can define a rank-$h$ truncation of the dataset $A^{(h)}$ as
\beq
\label{pod_truncation}
A_{i j}^{(h)} = \sum_{q=1}^h \sigma_q u_q(\vec{r}_i) w_q(t_j).
\eeq
What makes the POD more optimal than any other decomposition is that this truncation approximation is better than any other rank-$h$ approximations with other bases. Formally,
\beq
\label{pod_optimality}
\Vert A - A^{(h)} \Vert^2 = {\rm min} \left\{ \Vert A - B \Vert^2 \right\}, {\rm for \ rank}(B) = h.
\eeq
where $\Vert A \Vert = \sqrt{\sum_{i j} \Vert A_{i j} \Vert^2}$ is the $L_2$ norm.
In general, $\Vert A \Vert^2 = \sum_{q=1}^{N_{\rm{POD}}} \sigma_q^2 $ is the energy of the data and $\sigma_q^2$ represents the energy contained in the $q^{th}$ POD mode. In this sense, the POD is a
decomposition of the data in terms of energy content. In other words, the modes with the highest $\sigma_q$ comprise most of the energy of the data. If the $\sigma_q$'s descend rapidly, as they
often do, the truncated data reconstitution of Eq.~\ref{pod_truncation} represents the original data quite well. This is obviously useful in energetics analyses.

\subsection{POD Energy Dynamics}
\label{ss_pod_ed}

In order to construct the energy dynamics of the POD modes, I must first alter the data a bit because
energy of the system in Eq.~\ref{tot_energy} is not simply given by $\Vert A \Vert^2$ if $A$ is made up of the variables $(N, \phi, \vpe, T_e)$ from which I constructed the POD above.
The reason is that the energy
contains equilibrium constants as well as the perpendicular gradient of $\phi$ rather than $\phi$ itself; the energy is not simply $E \ne \Vert A \Vert^2 = \left( N^2 + \phi^2 + \vpe^2 + T_e^2 \right)$. 
To fix this, I instead reconstruct $A$ from the variables
$(\sqrt{T_{e0}/N_0} N, \sqrt{N_0} \grad_r \phi, \sqrt{N_0} \grad_\theta \phi, \sqrt{N_0} \fmei \vpe, \sqrt{3 N_0/2 T_{e0}} T_e)$. This variable weighting provides the equality, 
$E = \frac{1}{2} \Vert A \Vert^2$. I then perform the POD with this $A$.
The spatial POD modes $u_q$ can be unweighted, broken apart, and unraveled to get back functions such as $n_q(\vec{r})$ which is the density part of the $u_q$ POD mode. Let me give a name to this
unweighted vector: $x_q = (n_q, \phi_q, v_q, t_q)$. The original data is still decomposed in terms of this vector,
\beq
\label{unweighted_decomp}
(N,\phi,\vpe,T_e) = \sum_{q=1}^{N_{\rm{POD}}} \sigma_q x_q(\vec{r}) w_q(t)
\eeq
and equivalently
\beq
\label{n_pod_decomp}
N = \sum_{q=1}^{N_{\rm{POD}}} \sigma_q n_q(\vec{r}) w_q(t)
\eeq
but there is no orthogonality relation of these $n_q$ or even the full $x_q$ vectors, i.e.,
\beq
\label{x_nonorthogonality}
\sum_{i=1}^{N_{\rm{POD}}} n_q(\vec{r}_i) n^*_l(\vec{r}_i) \ne \delta_{q l}, \sum_{i=1}^{N_{\rm{POD}}} x_q(\vec{r}_i) x^*_l(\vec{r}_i) \ne \delta_{q l}.
\eeq
Note that I construct the data
that goes into the POD with 5 variables rather than 4 due to the need for both $-E_r = \grad_r \phi$ and $-E_\theta = \grad_\theta \phi$ in the energy expression and the requirement that $A$
be a scalar. This is straight-forward enough to do, but a potential complication arises when I unweight the $u_q$ POD modes to recover $\phi_q(\vec{r})$. Seemingly, the unweighting
recovers a $\phi_{r,q}$ and a $\phi_{\theta,q}$ that can be different, but in sum, $\sum_{q=1}^{N_{\rm{POD}}} \sigma_q \phi_{r,q} w_q = \sum_{q=1}^{N_{\rm{POD}}} \sigma_q \phi_{\theta,q} w_q = \phi$ must hold.
Fortunately, however, the POD ensures that $\phi_{r,q} = \phi_{\theta,q}$ because each of the POD modes preserves any mathematical property of the original data such as the zero-curl nature
of the electric field. I have confirmed this directly.

With this change of definition of $A$ and the corresponding definitions for the POD modes, I can now construct the POD energy dynamics. Like in the previous sections, I start with Eqs.~\ref{ni_eq}-
\ref{te_eq}. This time, I decompose the fields in terms of the POD modes. For example, I substitute $N(\vec{r},t) = \sum_{q=1}^{N_{\rm{POD}}} \sigma_q n_q(\vec{r}) w_q$ into $N$ in the equations,
and the same for the other independent variables $(\phi, \vpe, T_e)$. Eq.~\ref{ni_eq} then becomes
\beqar
\label{pod_ni_eq}
\sum_{q=1}^{N_{\rm{POD}}} \sigma_q n_q \pdiff{w_q}{t} & = & \sum_{q=1}^{N_{\rm{POD}}} \sigma_q w_q \left[- \frac{1}{r} \pdiff{\phi_q}{\theta} \pdiff{N_0}{r} - 
N_0 \pdiff{v_q}{z} + \mu_N \gradperp^2 n_q \right] + S_N \nonumber \\
& + & \frac{1}{r} \sum_{q,l} \sigma_q \sigma_l w_q w_l \left( \pdiff{\phi_q}{r} \pdiff{n_l}{\theta} - \pdiff{n_q}{r} \pdiff{\phi_l}{\theta} \right).
\eeqar
Next, I multiply this equation through by $T_{e0}/N_0 \sigma_p n^*_p w^*_p$ and the other equations by their corresponding energy prefactors and POD's. The LHS of the density equation is
\beq
\label{pod_ni_lhs}
\sum_{q=1}^{N_{\rm{POD}}} \frac{T_{e0}}{N_0} \sigma_q \sigma_p n_q n^*_p w^*_p \pdiff{w_q}{t} 
\eeq
In Sec.~\ref{s_spec_en_dyn}, volume integrating this term at this point isolated $E_N(\vec{k})$. Volume integration of Eq.~\ref{pod_ni_lhs} will not
produce the density energy of POD mode $p$: $E_N(p)$. The reason is that $n_q$ and $n^*_p$ are not orthogonal under volume integration as I pointed out in Eq.~\ref{x_nonorthogonality}. They are not
even orthogonal under volume integration with the appropriate energy prefactor. The reason is that
$n_q$ is only part of the total POD mode, which contains $\phi_q, v_q,$ and $t_q$ as well. The orthogonality relation only holds when the energy of all of these are summed and the prefactors
are added. That is, the orthogonality relation for the POD modes is a total energy orthogonality. As a consequence, there is no meaning to an equation for $\pdiff{E_N(p)}{t}$. There is only
meaning to an equation for the evolution of the energy of an entire POD mode: $\pdiff{E_{tot}(p)}{t}$. So, adding the four equations together and integrating over the volume results in
a LHS of
\beq
\label{pod_lhs_en}
\sigma_p^2 w^*_p \pdiff{w_p}{t} = \frac{1}{2} \pdiff{\left( \sigma_p^2 |w_p|^2 \right)}{t} = \pdiff{E(p)}{t}.
\eeq
The RHS, on the other hand, does not simplify much at all upon summing the equations together and performing volume integration -- other than the fact that the adiabatic response terms cancel each
other due to the equation summation. Take the $- {\bf v}_E \cdot \grad N_0$ term as an example.
As it stands, the term appears as
\beq
\label{pod_first_term}
- \int_V \left( \sum_{q=1}^{N_{\rm{POD}}} \frac{1}{r} \sigma_q \sigma_p w_q w^*_p n^*_p  \pdiff{\phi_q}{\theta} \pdiff{N_0}{r} \right) dV.
\eeq
While such a term may be calculated as is, it is unfortunate that the sum over all of the POD modes remains. However, recall that the temporal $w_q$ POD modes are orthogonal to each other upon
time integration (Eq.~\ref{pod_orthonormality}). So if I time integrate the energy evolution equation, this term becomes
\beq
\label{pod_first_tint}
- \int_V \left( \frac{T_{e0}}{r N_0} \sigma_p^2 n^*_p  \pdiff{\phi_p}{\theta} \pdiff{N_0}{r} \right) dV,
\eeq
which is only a function of a single POD mode! This step is one of the reasons why the POD is preferable to other decompositions like the left/right linear eigenmode decomposition. One may notice,
however, that time integration trivializes the LHS $\pdiff{E(p)}{t}$, which becomes approximately zero in the steady-state regime. 
Necessarily, the RHS must be zero as well. However, the goal is to understand the mode dynamics in the
steady-state regime, and simply separating linear and nonlinear terms on the RHS, like what I did in Eq.~\ref{Fourier_density_evolution} can give information regarding which POD modes inject
energy into the system, which ones dissipate energy, and which modes transfer to which other modes. Specifically,
\beq
\label{pod_ev_breakdown}
\int_t \pdiff{E(p)}{t} dt = L(p) + \sum_{q,l} T(p,q,l) = 0
\eeq
with the $L(p)/\sigma_p^2$ being the volume integral of
\beqar
\label{L_pod}
& & \frac{T_{e0}}{N_0} n^*_p \left( - \frac{1}{r} \pdiff{\phi_p}{\theta} \pdiff{N_0}{r} + \mu_N \gradperp^2 n_p \right) \nonumber \\
& + & \frac{3 N_0}{2 T_{e0}} t^*_p \left( - \frac{1}{r} \pdiff{\phi_p}{\theta} \pdiff{T_{e0}}{r} + \mu_T \gradperp^2 t_p + \frac{2 \kappa_{\para e}}{3 N_0} \gradpar^2 t_p - \frac{2 m_e}{m_i} \nue t_p  \right) \nonumber \\
& - & \phi_p^* \left( \mu_\phi \gradperp^2 \rho_p + \nuin \rho_p \right) - \nue \fmei N_0 |v_p|^2
\eeqar
where positive values of $L(p)$ indicate energy injection into the POD $p$-mode from the equilibrium gradients, while negative values indicate dissipation of energy. $L(p)$ of course, is the
linear nonconservative part of the energy dynamics.
The conservative nonlinear transfer term, unfortunately, cannot be simplified with the orthogonality relations 
due to the appearance of triple time products ($\int_t w_q w_l w^*_p dt$) and spatial products without
the right form for orthogonality. I do not write it out explicitly here, but I note that $T(p,q,l)$ represents the energy transfer from POD modes $q$ and $l$ to POD mode $p$. To reiterate,
Eq.~\ref{L_pod} determines which POD modes inject energy into the fluctuation system, which ones dissipate energy, and how much they do so.
