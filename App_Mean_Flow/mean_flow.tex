\chapter{Finite Mean Flow Simulations}
\label{app_mean_flow}

In the main text, I focused on one particular LAPD experiment, which essentially contained no mean ${\bf E \times B}$ flow or flow shear. Focusing on this experiment
allowed me to model the system with a smaller number of linear terms in the equation set than if the experiment had contained significant ${\bf E \times B}$ flow. 
Furthermore, neglecting mean flow and flow shear eliminated a couple of linear instabilities, namely Kelvin-Helmholtz
and rotational interchange instabilities, which are both flute-like ($k_{\para} = 0$) instabilities. It would have been less obvious, then, that the nonlinear instability caused the appearance
of dominant $k_{\para} = 0$ structures, though ultimately, energy analysis would have sorted this out. Furthermore, the low flow experiments have proven to be easier to successfully simulate than
the high flow experiments, and the low flow experiment and simulations contain so much interesting physics that they deserve study in their own right.

The high ${\bf E \times B}$ flow experiments, which also have high flow shear, are quite relevant to tokamak research, specifically research into the High Confinement Mode (H-mode)
and the transition to the H-mode. Researchers have long realized that H-mode is associated with strong toroidal rotation of the tokamak plasma and that the shear associated with this rotation
is the likely cause of of the decrease in energy transport. The particular physical mechanism of turbulent-shear interaction that causes the flux suppression is still an area of intense
research, and many testable theories have been created to address this.


\section{The LAPD Biasing Experiment}
\label{s_biasing_exp}

\section{New Linear Instabilities}
\label{s_flow_inst}

\beqar
\label{ni_eq_flow}
\pdt N = - {\mathbf v_E} \cdot \grad N_0 - {\mathbf v_{E0}} \cdot \grad N - N_0 \gradpar \vpe + \mu_N \gradperp^2 N + S_N + \{\phi,N\}, \\
\label{ve_eq_flow}
\pdt \vpe = - {\mathbf v_{E0}} \cdot \grad \vpe - \fmie \frac{T_{e0}}{N_0} \gradpar N - 1.71 \fmie \gradpar T_e + \fmie \gradpar \phi - \nue \vpe + \{\phi,\vpe \}, \\
\label{rho_eq_flow}
\pdt \varpi = - {\mathbf v_E} \cdot \grad \varpi_0 - {\mathbf v_{E0}} \cdot \grad \varpi- 
\frac{1}{r} \pdiff{\phi_0}{r} \left(\pdiff{N_0}{r} \pdiffxy{\phi}{r}{\theta} - \pdiffs{\phi_0}{r} \pdiff{N}{\theta} \right) \nonumber \\
 - N_0 \gradpar \vpe  - \nuin \varpi + \mu_\phi \gradperp^2 \varpi + \{\phi,\varpi \}, \\
\label{te_eq_flow}
\pdt T_e = - {\mathbf v_E} \cdot \grad T_{e0} - {\mathbf v_{E0}} \cdot \grad T_e - 1.71 \frac{2}{3} T_{e0} \gradpar \vpe + \frac{2}{3 N_0} \kpe \gradpar^2 T_e  \nonumber \\
- \frac{2 m_e}{m_i} \nue T_e  + \mu_T \gradperp^2 T_e +  S_T + \{\phi,T_e\}.
\eeqar


\section{Statistical Comparisons to Experiment}
\label{s_flow_stats}

\section{Energy Dynamics Results}
\label{s_flow_en_dyn}
