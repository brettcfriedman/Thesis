\title          {Numerical Studies of Turbulence in LAPD}
\author         {Brett Cory Friedman}

\department     {Physics}
\degreeyear     {2013}

\chair          {Troy A. Carter}
\member         {Russel E. Caflisch}
\member         {George J. Morales}
\member         {Warren B. Mori}

%%%%%%%%%%%%%%%%%%%%%%%%%%%%%%%%%%%%%%%%%%%%%%%%%%%%%%%%%%%%%%%%%%%%%%%%
%% Dedication

\dedication     {\textsl{To my parents \ldots \\}}
                
%%%%%%%%%%%%%%%%%%%%%%%%%%%%%%%%%%%%%%%%%%%%%%%%%%%%%%%%%%%%%%%%%%%%%%%%
%% Acknowledgements

\acknowledgments {This thesis culminates several years of hard work, and could not have been completed without the help of many individuals to whom I owe much thanks. First, I must thank my
advisor, Professor Troy Carter, for giving me enough guidance throughout the process to help me constantly progress, and more importantly, for allowing me great freedom in following whatever
courses of research I deemed most interesting. I could not have asked for a more knowledgeable, understanding, and helpful advisor and friend. Also, I thank Dr. Maxim Umansky
from LLNL, who has acted as a mentor and a second advisor throughout my time at UCLA. While under no obligation to help, he has continuously given advice, support, knowledge, direct help,
and friendship throughout the process. I would not have made nearly as much progress without him, and I greatly appreciate his mentorship while I briefly worked at Livermore. Furthermore,
Dr. Pavel Popovich helped me start on this research, being increadibly patient while I tried to learn everything from Linux commands to plasma theory, asking him nearly every day a number
of basic questions that he happily answered. He helped ease much of the frustration associated with starting a new field of research, and I owe him so much thanks.

I would also like to thank UCLA Professor George Morales for teaching me the basics of plasma physics. His methods instilled in me an appreciation for mathematical rigor, careful analysis, and the
importance of in depth reading of papers. He has also made himself available to me for research questions and explanations of difficult theoretical topics. I thank Dr. James Maggs for helping
explain to me concepts of chaos and stochasticity and providing me with code and information in order to fill out my chapter on chaos. Additionally, I acknowledge and thank
York Professor Ben Dudson for writing the fantastic BOUT++ code, which has made my work so much easier. I also thank him for quickly and effectively answering all of my questions regarding the code,
and for fixing some bugs that could have set me back a long time. 

I owe much appreciation to Professor Paul Terry of the University of Wisconsin for many useful discussions on a number of topics. He always pointed me in the right direction and often provided
great insight into my results. Furthermore Dr. David Hatch spent a lot of time working with me on a proper orthogonal decomposition for my simulations, even writing code to decompose my data.
Additionally, LLNL Dr. Ilon Joseph deserves thanks for carefully reading my work, discussing theory with me, teaching me a lot,
and most importantly for suggesting a change to one of my equations that has greatly improved my work. Also, I thank LLNL Dr. Xueqiao Xu for our many helpful discussions, for helping create the
BOUT++ community and bringing me into it, and for his help in advancing my career goals. Finally, Dr. David Schaffner has always provided me with all of the experimental data I could ever ask for and
more, never making me wait.

I also must acknowledge the great friends I have made during my time at UCLA. Without them, I probably would have made much more progress and finished faster, but then again, I may not have made it
out alive without them. Brandon, Scott, Dan, other Scott -- their friendship has been invaluable. David, Tom, Derek, Erik, Gio, Seth, Mike, Yuhou, Danny -- my plasma buddies, conference friends, people
who let me ramble on about theory; I thank you all. I thank Eric and Mikhail for taking me in during my time at LLNL, making that experience truly enjoyable. I thank so many others at UCLA.
And, of course, I thank my absolutely wonderful girlfriend, Alli for making my life so much better.

And last but not least, I thank my entire family, all of whom have always supported me, loved me, encouraged me, and raised me to be who I am today. I could not ask for better parents or a better
sister. And I especially acknowledge my grandparents, who have given me all of the encouragement, love, and sustenance they could over the past several years.}


%%%%%%%%%%%%%%%%%%%%%%%%%%%%%%%%%%%%%%%%%%%%%%%%%%%%%%%%%%%%%%%%%%%%%%%%
%% Vita

\vitaitem   {2003-2007}
                {Regent Scholar, University of California, Irvine.}

\vitaitem   {2007}
                {B.S.~(Physics) Summa Cum Laude, UC Irvine.}

\vitaitem   {2007}
                {Chancellor's Fellow, University of California, Los Angeles.}

\vitaitem   {2007-2009}
                {Teaching Assistant, University of California, Los Angeles.}

\vitaitem   {2009-2013}
                {Research Assistant, University of California, Los Angeles.}

\vitaitem   {2009-2012}
                {ORISE FES Fellow, University of California, Los Angeles.}

\vitaitem   {2012-2013}
                {Dissertation Year Fellow, University of California, Los Angeles.}


%%%%%%%%%%%%%%%%%%%%%%%%%%%%%%%%%%%%%%%%%%%%%%%%%%%%%%%%%%%%%%%%%%%%%%%%
%% Publications
\publication{B. Friedman, T. A. Carter, M. V. Umansky, D. Schaffner, and I. Joseph, Nonlinear instability in simulations of Large Plasma Device turbulence, Phys. Plasmas 20, 055704 (2013).}

\publication{D. A. Schaffner, T. A. Carter, G. D. Rossi, D. S. Guice, J. E. Maggs, S. Vincena, and B. Friedman, Turbulence and transport suppression scaling with flow shear on the Large Plasma Device, Phys. Plasmas 20, 055907 (2013).} 

\publication{B. Friedman, T. A. Carter, M. V. Umansky, D. Schaffner, and B. Dudson, Energy dynamics in a simulation of LAPD turbulence, Phys. Plasmas 19, 102307 (2012).}

\publication{D. A. Schaffner, T. A Carter, G. D. Rossi, D. S. Guice, J. E. Maggs, S. Vincena, and B. Friedman, Modification of Turbulent Transport with Continuous Variation of Flow Shear in the Large Plasma Device, Phys. Rev. Lett. 109, 135002 (2012).}

\publication{S. Zhou, W. W. Heidbrink, H. Boehmer, R. McWilliams, T. A. Carter, S. Vincena, B. Friedman, and D. Schaffner, Sheared-flow induced confinement transition in a linear magnetized plasma, Phys. Plasmas 19, 012116 (2012).}

\publication{B. Friedman, M. V. Umansky, and T. A. Carter, Grid convergence study in a simulation of LAPD turbulence, Contrib. Plasma Phys. 52, 412 (2012).}

\publication{M. V. Umansky, P. Popovich, T. A. Carter, B. Friedman, and W. M. Nevins, Numerical simulation and analysis of plasma turbulence the Large Plasma Device, Phys. Plasmas 18, 055709 (2011).}

\publication{P. Popovich, M.V. Umansky, T.A. Carter, and B. Friedman, Modeling plasma turbulence and transport in the Large Plasma Device, Phys. Plasmas 17, 122312 (2010).}

\publication{P. Popovich, M.V. Umansky, T.A. Carter, and B. Friedman, Analysis of plasma instabilities and verification of the BOUT code for the Large Plasma Device, Phys. Plasmas 17, 102107 (2010).}

\publication{S. Zhou, W. W. Heidbrink, H. Boehmer, R. McWilliams, T. A. Carter, S. Vincena, S. K. P. Tripathi, P. Popovich, B. Friedman, and F. Jenko, Turbulent transport of fast ions in the Large Plasma Device, Phys. Plasmas 17, 092103 (2010).}

%%%%%%%%%%%%%%%%%%%%%%%%%%%%%%%%%%%%%%%%%%%%%%%%%%%%%%%%%%%%%%%%%%%%%%%%
%% Abstract

\abstract{I model, simulate, and analyze the turbulence in a particular experiment on the Large Plasma Device (LAPD) at UCLA.
The model is an electrostatic reduced Braginskii two-fluid model that describes the time evolution
of density, electron temperature, electrostatic potential, and parallel electron velocity fluctuations in the edge region of LAPD. The spatial domain is annular, encompassing the radial coordinates
over which a significant equilibrium density gradient exists.
My model breaks the independent variables in the equations into time-independent equilibrium parts and time-dependent fluctuating parts, and I use experimentally obtained values as input for the
equilibrium parts.

After an initial exponential growth period due to a linear drift wave instability, the fluctuations saturate and the frequency and azimuthal wavenumber spectra become broadband
with no visible coherent peaks, at which point the fluctuations become turbulent. The turbulence develops intermittent pressure and flow filamentary structures that grow and dissipate, but look much
different than the unstable linear drift waves, primarily in the extremely long axial wavelengths that the filaments possess. An energy
dynamics analysis that I derive reveals the mechanism that drives these structures. The long $k_\para \sim 0$ intermittent potential 
filaments convect equilibrium density
across the equilibrium density gradient, setting up local density filaments. These density filaments, also with $k_\para \sim 0$, produce azimuthal density gradients, which drive radially
propagating secondary drift waves. These finite $k_\para$ drift waves nonlinearly couple to one another and reinforce the original convective filament, allowing the process to bootstrap itself. The growth
of these structures is by nonlinear instability because they require a finite amplitude to start, and they require nonlinear terms in the equations to sustain their growth.

The reason why $k_\para \sim 0$ structures can grow and support themselves in a dynamical system with no $k_\para = 0$ linear instability is because the linear eigenmodes of the
system are nonorthogonal. Nonorthogonal eigenmodes that individually decay under linear dynamics can transiently inject energy into the system, allowing for instability.
The instability, however, can only occur when the fluctuations have a finite starting amplitude, and nonlinearities are available to mix energy among eigenmodes.

Finally, I attempt to figure out how many effective degrees of freedom control the turbulence to determine whether it is stochastic or deterministic. 
Using two different methods -- permutation entropy analysis by means of time delay trajectory reconstruction and Proper Orthogonal Decomposition -- I determine that
more than a few degrees of freedom, possibly even dozens or hundreds, are all active. The turbulence, therefore, is not a manifestation of low-dimensional chaos -- it is stochastic.
}
