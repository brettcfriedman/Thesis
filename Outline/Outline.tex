\documentclass[12pt]{article}
\usepackage{graphicx}
\setlength{\parindent}{0in}


\begin{document}

{\bf \large Thesis Outline \\ \\}

Author: Brett Friedman \\
Updated: \today \\

%%%%%%%%%%%%%%%%%%%%%%%%%%%%%%%%%%%%%%%%%%%
\section{Introduction}

- The problem of anomalous transport in tokamaks \\
- Understanding the nature of turbulence in LAPD \\
- The use of simulation to acquire spatial data and theoretical understanding \\
- The nonlinear instability in LAPD \\

%%%%%%%%%%%%%%%%%%%%%%%%%%%%%%%%%%%%%%%%%%%
\section{Turbulence and Instability}

\subsection{The Kolmogorov Paradigm of Turbulence}

- High Reynolds number neutral fluid turbulence \\
- Energy injection, cascading, and dissipation \\
- \emph{Figure}: Standard cascading figure \\

\subsection{The Standard Plasma Paradigm of Linear Instability}

- The plethora of linear instabilities in plasma physics \\
- The dual cascade \\
- The problem of low Reynolds number, lack of inertial range, etc. in plasmas \\
- The reliance on linear theory for understanding and quasi-linear calculations \\

\subsection{Nonlinear Instability: Subcritical and Supercritical Instability}

- Subcritical turbulence in pipe (Poiselle) flow \\
- \emph{Figure}: Pitchfork diagrams and explanation of sub and supercritical instability \\
- Note that other kinds of nonlinear instabilities exist \\

\subsection{Nonlinear Instability in Plasmas}

- ITG (Dimits shift) in GK Simulations \\
- TEM (Ernst) GK Simulations \\
- Various subcritical plasma instabilities \\
- Bruce Scott self-organization findings \\
- Drake, Biskamp, Zeiler papers \\

%%%%%%%%%%%%%%%%%%%%%%%%%%%%%%%%%%%%%%%%%%%
\section{The Braginskii Fluid Model and LAPD}

\subsection{LAPD Suitability to Fluid Model}

- High collisionality (collision mfp over the machine length) and neglect of Landau damping \\
- FLR effects \\

\subsection{Braginskii Equations}

- Origin of the equations \\
- Assumptions and orderings \\
- Write the full equations (Tokamaks book?) \\
- Describe the meaning of the terms \\

\subsection{The Vorticity Equation}

- Justify the quasi-neutrality condition \\
- Derive the Vorticity Equation \\

\subsection{Minimizing the Equation Set for LAPD parameters}

- Neglect of ion temperature and parallel velocity fluctuations \\
- The electrostatic justification \\
- Make a comparison of the collision to mass to inductive effects on the adiabatic response (see B.D. Scott paper) \\
- \emph{Figure}: Beta scans comparing ES vs EM growth rates \\

%%%%%%%%%%%%%%%%%%%%%%%%%%%%%%%%%%%%%%%%%%%
\section{The BOUT++ Code}

\subsection{The Object-Oriented Fluid Framework}

- The original BOUT \\
- The advantage of BOUT++ \\
- Parallel framework \\

\subsection{Explicit Finite Differences}

- The time solver options and the one I use \\
- The spatial finite difference schemes \\

\subsection{The Physics Inputs}

- Put lapd-drift.cxx in an appendix \\
- The grid file \\
- The input file \\

%%%%%%%%%%%%%%%%%%%%%%%%%%%%%%%%%%%%%%%%%%%
\section{LAPD Simulation Details}

\subsection{The Equations}

- Partially linearized equations (with $\phi_0$?) \\

\subsection{Finite Difference Schemes}

- Quasi-staggered method \\
- Advection schemes and positivity \\

\subsection{Artificial Diffusion and Grid Convergence}

- Grid convergence paper \\
- \emph{Figure}: U1 advection figures in the paper \\
- \emph{Figure}: Arakawa advection figures in the paper \\

\subsection{Sources}

- \emph{Figure}: Relaxation without the sources caused by transport \\
- Simple density and temperature sources (removal of azimuthal average) \\
- Positivity-preserving sources \\

\subsection{Boundary Conditions}

- The annular geometry and justification of zero-value radial boundaries \\
- Periodic, zero-value, and zero-derivative axial boundaries \\
- Sheath boundary condition and specific implementation \\

\subsection{Profiles and Parameters}

- \emph{Figure}: Density and temperature profiles \\
- Zero mean potential profile justified by biasing experiment \\

%%%%%%%%%%%%%%%%%%%%%%%%%%%%%%%%%%%%%%%%%%%
\section{Linear Instabilities}

\subsection{Drift waves}

- Linearized equations \\
- \emph{Figure}: Physical mechanism \\
- \emph{Figure}: Growth rates with periodic, zero-value and zero-derivative boundary conditions \\

\subsection{Conducting Wall Mode}

- Physical mechanism \\
- The reduced fluid equations used in BOUT++ \\
- Specific implementation of the boundary condition \\
- \emph{Figure}: Growth rates with comparison to drift waves \\


%%%%%%%%%%%%%%%%%%%%%%%%%%%%%%%%%%%%%%%%%%%
\section{The Nature of the Turbulence}

\subsection{A Visual Examination}

- \emph{Figure}: 2D and 3D movies of the turbulence \\
- \emph{Figure}: Movie comparison to camera data \\
- Note the formation of long parallel wavelength structures \\

\subsection{A Statistical Examination}

- \emph{Figure}: Spectra, PDFs, Fluctuation Levels \\
- Note that the experimental and simulation frequency spectra are not exponential \\
- \emph{Figure}: Correlation Plots: simulation and exp. \\


%%%%%%%%%%%%%%%%%%%%%%%%%%%%%%%%%%%%%%%%%%%
\section{Energy Dynamics Formalism}

\subsection{Total Energy and Dynamics}

- State the physical energy \\
- Derive the full energy evolution \\
- Discuss the meaning of the pieces (sources, transfers, etc.) \\
- Discuss the conservation (and lack of it) for the advective nonlinearities with reference to first paper \\

\subsection{Spectral Energy Dynamics}

- Derive the spectral density energy evolution \\
- State the results for the other energy expressions \\

%%%%%%%%%%%%%%%%%%%%%%%%%%%%%%%%%%%%%%%%%%%
\section{Nonlinear Instability for the Periodic Simulation}

\subsection{The Energy Spectra}

- \emph{Figure}: The energy spectra
- The energetic dominance of $k_\parallel = 0$ fluctuations \\
- Our original misunderstanding of the $k_\parallel = 0$ structure origins due to the standard linear instability picture \\

\subsection{Energy Dynamics Result}

- \emph{Figure}: The full detailed dynamics from first paper \\
- \emph{Figure}: The full diagram from the second paper \\
- \emph{Figure}: The reduced diagram from the first paper \\
- \emph{Figure}: The growth rate, linear vs nonlinear \\

\subsection{n=0 Suppression}

- \emph{Figure}: The coherent turbulence (movie or snapshot) 
- \emph{Figure}: The statistical difference \\
- \emph{Figure}: The energy dynamics and growth rates \\
- \emph{Figure}: The role of zonal flows \\

%%%%%%%%%%%%%%%%%%%%%%%%%%%%%%%%%%%%%%%%%%%
\section{Energy Dynamics for the Non-periodic Simulations}

\subsection{The Importance of Axial Boundary Conditions}

- Discuss the need to explore different boundary conditions with reference to linear growth rate curves \\
- \emph{Figure}: The statistics of the different cases \\

\subsection{Fourier Decomposing Non-periodic Functions}

- \emph{Figure}: The power law vs exponential convergence \\
- The appendix of the second paper \\

\subsection{Energy Dynamics Results}

- \emph{Figure}: Energy flow diagrams (at least for the sheath case, maybe one more) \\
- Growth rate plots: linear vs nonlinear \\

\subsection{Linear vs Nonlinear Structure Correlation}

- Section from the second paper \\
- \emph{Figure}: Ratio of fastest growing mode energy to total energy \\

%%%%%%%%%%%%%%%%%%%%%%%%%%%%%%%%%%%%%%%%%%%
\section{The Finite Mean Flow Simulations}

\subsection{The Biasing Experiment}

- Shear suppression paradigm \\
- Simulations not focusing on dynamical profile evolution yet \\

\subsection{New Linear Instabilities}

- KH \\
- Rotational Interchange \\
- \emph{Figure}: Isolation of KH and IC instabilities

\subsection{Statistical Comparisons to Experiment}

- \emph{Figure}: Unbiased, medium and high biases \\
- Note the skewed pdfs due to the radial force and blob production \\

\subsection{Energy Dynamics Results}

- The new terms in the energy dynamics and their meaning \\
- Instability drive comparisons \\
- \emph{Figure}: Energy flow diagram \\
- The coherent mode? \\

%%%%%%%%%%%%%%%%%%%%%%%%%%%%%%%%%%%%%%%%%%%
\section{Conclusion}


\section{Appendix 1}

- lapd-drift.cxx code

\end{document}
