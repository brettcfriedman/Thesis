\chapter{Introduction}
\label{c_intro}

\section{Motivation}
\label{s_motivation}

Thermonuclear fusion has the potential to solve the world's energy problems. The fusion of a pair of light nuclei, such as deuterium and tritium, releases more energy than the fission
of a uranium nucleus and much more energy than the chemical reactions involved in the burning of fossil fuels. And the fusion products 
-- unlike those from nuclear fission and fossil fuel burning-- are relatively harmless to the environment. Furthermore, fusion fuel sources are much more abundant than fossil fuels
and fissionable uranium. The limiting component of dueterium-tritium fusion reactions is the tritium, which can be made from lithium. But there is enough lithium on Earth to power the world
through nuclear fusion for at least a million years~\cite{wesson2004}.

Due to its great potential, scientists have been working on controlling fusion reactions for over half a century. The community has made much progress, but fusion is not yet a commercially
viable energy source, and there are still several scientific and technological obstacles to overcome before it is. The main obstacle to achieving controlled fusion reactions the confinement of the
reaction fuel for a long enough time at sufficiently high temperatures and densities to achieve a self-sustaining reaction. 
One of the best ways to contain the fuel is to maintain it in a plasma state and restrict the motion of the plasma using magnetic fields. 
Possibly the best way to do this -- and certainly the most intensely studied -- is with a tokamak. Tokamaks around the world have already achieved temperatures and pressures
necessary to produce fusion but have yet to confine the plasmas for long enough to achieve a self-sustained fusion reaction. Seemingly, this problem can be solved by making
the tokamaks larger, although it's still unclear if the large tokamaks will be able to operate in a high confinement mode (H-mode), which will probably be necessary in order to
keep the tokamaks economically viable. It's also unclear if the material walls of the larger tokamaks will be able to survive the large plasma fluxes.

The plasma confinement time, for a given sized tokamak, is inversely proportional to the rate of cross-field transport in the tokamak, so it's important to know how to minimize that
transport. Now, while the energy transport must be minimized, fusion product alpha particles (and other non-intrinsic impurities) must be transported out of the core so particle transport
must be kept sufficiently high. The cross-field particle and energy transport is primarily driven by microturbulence, which is driven by instabilities due to the presence of free energy sources.
These free energy sources are due to non-Maxwellian velocity-space features of the distribution functions, spatial inhomogeneity of the distribution functions, and stored electromagnetic energy.
These energy sources always exist in the normal operating conditions of a tokamak. For example temperature gradients must exist in tokamaks
since the hot tokamak core cannot extend all the way to the walls, which are kept close to room temperature. In fact, we must take great care in order to prevent a high flux of extremely
hot plasma from hitting the material walls, which can melt them or sputter atoms into the core. 
Transport can actually help in this regard by spreading out the plasma beam that crosses the last-closed-flux-surface so that its flux
per unit area hitting material limiters and divertor targets is reduced. Altogether, turbulent transport is needed in some ways, but is detrimental in other ways. A balance may be key,
or maybe clever techniques and engineering can be used to control the transport in the necessary ways. In either case, it's important to be able to predict how the turbulence and the
transport will react to changes in design or changes in operational parameters. 

Predicting transport has been a long, slow research activity for some time. One large problem is that turbulence is not completely understood in even the simplest cases of neutral fluids,
let alone the more complex cases of tokamak plasmas. Nevertheless, at this point, many aspects of turbulence and transport in the tokamak core are fairly well-understood, largely
due to the success of gyro-kinetic simulations. Turbulence in the edge, on the other hand, is not as well-understood for several reasons. One, the edge region contains complex magnetic
field geometry, where the field lines range from open to closed, the open ones ending on material surfaces. Two, turbulent fluctuations are high, invalidating current forms of the
gyro-kinetic equations, leaving no model to absolutely apply to the entire edge region. Three, the edge contains a zoo of potential instabilities that can drive the turbulence,
and seemingly different instabilities exist in different tokamaks and in different operating regimes.

Numerical simulations have helped improve understanding of physical processes and spatial and temporal structures in all kinds of turbulent settings. However,
experimental observations and analytic theory generally lead tokamak research, with simulations merely trying to confirm the ideas obtained from these more established methods.
Nevertheless, simulations can produce more detailed results than analytic theory and more spatial information than experimental observation, making them valuable.
Furthermore, the hope is that simulations can lead experiment, providing predictions before experiments are done, or at least providing enough physical insight to inform experimental tasks.
In some instances, like in the cores of tokamaks, the community has made enough progress on simulations (specifically gyro-kinetic ITG simulations) that simulations have uncovered new
unexpected physics (such as the Dimits shift of ITG turbulence~\cite{dimits2000}). But in other instances, like in the edge of tokamaks, nonlinear turbulent simulations don't yet
agree enough with experiment to provide good physical insight, let alone prediction capabilities for the reasons listed above. A possible path to solving this problem is to reduce the problem
to a simpler one, achieve simulation validation with that, and then slowly move up to more and more complex situations. A natural place to start is simulation of linear plasma devices.

Magnetic plasma devices that are simpler and colder than tokamaks, like the Large Plasma Device (LAPD) at UCLA~\cite{Gekelman1991}, have long been used to study basic plasma processes that
are relevant to tokamaks. These machines, which generally produce plasma turbulence, offer a more accessible environment to conduct experiments than a tokamak. They are also easier to
understand due to their relative simplicity, especially with regard to their magnetic field configurations, which also reduces the number of instabilities present in them. 
Furthermore, they are colder and thus more collisional than tokamaks, making fluid equations more applicable than they are in tokamaks. It should be easier to produce a verified, validated
simulation of turbulence in a linear machine like LAPD than in a tokamak. And any insight gained from analysis of the simulation may apply to edge tokamak turbulence as well, or at least
provide methods of analysis or ideas that may be checked when tokamak simulations become more successful.

\section{Dissertation Summary}
\label{s_summary}

My dissertation focuses on direct numerical simulations of low frequency turbulent fluctuations in LAPD. Since LAPD has low temperature ($T_e \le 10$ eV and $T_i \le 1$ eV) and is very long,
it is highly collisional, making it suitable for modelling with fluid equations. Thus, I use a reduced Braginskii two-fluid model~\cite{Braginskii1965} for the equations, which I expand around
an equilibrium, time-independent state (a Reynolds decomposition). 
I take the equilibrium density, electron temperature, magnetic field, and electrostatic potential profiles from experimental measurements. I use cold ions and neglect ion acoustic
wave effects. I perform both linear and nonlinear simulations with the BOUT++ code~\cite{dudson2009}, where I retain only the advective nonlinearities in the nonlinear simulations.
In the nonlinear turbulent simulations, I apply corrective density and temperature sources to keep the flux-surface averaged density and temperature from diverging from the quasi-steady state
experimental profiles.

In all but the last chapter of the dissertation, I focus on simulating a single experimental configuration. In this configuration, the LAPD plasma is created solely with the cathode/annode
source that emits and accelerates electrons, which ionize neutral helium atoms. The only obstructing feature is an azimuthal limiter about 30 cm in radius, which is close to the annode. 
This limiter is also biased so that the equilibrium radial electric field
is nulled out. This simplifies the simulations because it eliminates one of the equilibrium profiles that can cause instability. Furthermore, it eliminates any questions regarding
${\bf E \times B}$ shear-induced turbulent suppression, which is an important field of study in its own right but not my focus.
So the only equilibrium gradients in this experiment and in the simulations are the density and electron temperature gradients, which point in the radial direction and are strongest around the limiter edge.
In the last chapter before the conclusion, I simulate experiments in which the limiter is biased differently, so that different non-zero radial electric fields exist. I still focus, however, on the instability
properties rather than on the ${\bf E \times B}$ shear dynamics.

In the null radial electric field simulations, I use several different idealized axial boundary conditions including periodic, zero-value, zero-derivative, and Bohm sheath. For the first three
axial boundary conditions, the only linear instability that exists in the model is the drift wave instability, which can be driven by both the equilibrium density and electron temperature gradients.
For the simulation with the Bohm sheath boundary condition, a Conducting Wall Mode Instability also exists, and its growth rate is comparable to the drift wave growth rate. However, while the linear
drift waves have finite $k_\para$ such that their axial wavelengths are on the order of the machine length (or twice the machine length), the linear Conducting Wall Mode waves have $k_\para \approx 0$.

I start the nonlinear null radial electric field simulations from small random noise. The fluctuations grow through their linear instabilities until they saturate with 
normalized density $\tilde{n}/n$ and potential $e \tilde{\phi}/T_e$ fluctuations reaching an RMS value of about $10 \%$. The saturation is due to nonlinear three-wave transfer from the unstable waves
to stable ones, where energy is ultimately dissipated by artificial diffusion and viscosity as well as ion-neutral collisions. 
The zonal flow may aid in this nonlinear transfer, but it doesn't seem to be a major player in the turbulent
dynamics, and I don't focus on it. In fact, I don't focus on the saturation mechanism or the dissipation routes at all.

Statistical analysis of the turbulence reveals that it is broadband and qualitatively and quantitatively similar to the turbulence in the experiment. The level of validation arguably means that
the results from the simulations can provide information regarding the fluctuations in the experiment. The simulations are useful because they provide a determinstic dynamical system to
associate with the experimental turbulence, and they provide much more spatial information than experimental measurements do. One immediate result that I obtain because of this enhanced spatial knowledge
is that the simulation turbulence is dominated by $k_\para = 0$ flute-like structures. I explore the cause and effects of this by deriving and analyzing the energy dynamics of the turbulence,
which is possible because of the deterministic dynamical systems aspect of the simulations.

The energy dynamics reveals dynamical processes such as energy injection from the free energy equilibrium gradients into the fluctuations, energy transfer between different waves or modes,
energy transfer between potential and kinetic energy degrees of freedom, and energy dissipation of the fluctuations. The most interesting result is that a multi-step nonlinear instability
process drives energy into the fluctuations, ultimately sustaining the turbulence. The process begins with $k_\para = 0$ convective cells advecting density across the equilibrium density gradient,
which build up $k_\para = 0$ density fluctuations. These density fluctuations break up by three-wave transfer (or equivalently they excite a secondary instability), driving finite $k_\para$
drift waves. These finite $k_\para$ drift waves have access to the adiabatic response, meaning they can transfer energy between potential and kinetic energy. The resultant drift waves then
transfer some of their kinetic energy back to the convective cells. This process is self-sustaining and it is necessarily nonlinear because the three-wave transfers from the $k_\para = 0$ density fluctuations
to the drift waves and from the drift waves back to the convective cells are both purely nonlinear processes. 

Now despite the nonlinear nature of the process, linear effects are still quite important due to the fact that the nonlinearities of the system are energetically conservative.
This means that the energy that supports the process ultimately comes from a linear mechanism, specifically the mechanism in which the convective cells advect density across the gradient. 
But if one thinks in terms of linear eigenmodes and linear instabilities, this probably doesn't fit with one's intuition. 
The reason is that the linear mechanism occurs at $k_\para = 0$, where no linearly unstable drift
waves exist, yet a linear process at $k_\para = 0$ still brings energy into the turbulent system. The nonintuitive process responsible for this is a transient growth mechanism of nonorthogonal
stable linear eigenmodes. Because the linear system is non-normal, the linear eigenmodes are not orthogonal to one another, and they might even be largely anti-parallel to each other.
When this happens, even when all of the eigenmodes decay, the system as a whole may still grow, although only transiently. Nevertheless, this growth, when reinforced by nonlinear
effects, can sustain itself, continuing to drive energy into the system. Interestingly, the nonlinear instability and the linear mechanism that drives it are analagous to those which drive turbulence
in many subcritical neutral fluid flows. This connection between neutral fluid and plasma turbulence may provide further hints on how to proceed with research in the future because the neutral
fluid community has worked so much with systems that are dominated by this instability process.

While the nonlinear instability is easy to identify in the simulation with periodic axial boundary conditions, it is not as easy to identify in the simulations with the other boundary conditions. The reason is
that the unstable linear eigenmodes of the simulations with non-periodic boundary conditions can have more flute-like structure than those of the periodic simulation. The simulation with Bohm
sheath boundaries is especially difficult to analyze because the linear eigenmodes are quite flute-like. Nevertheless, the similarity between these simulations and the periodic one combined
with energy dynamics evidence points to the fact that all of the simulations are dominated by the nonlinear instability mechanism, making it rather robust.

